\subsection{Direct analysis of an alt and channels}
\label{sec:combined}

\inlineCSP

We now describe how we can perform a direct analysis of an alt together with
channels.  

We build a system that uses an alt with a fixed collection of branches.
Below, we consider an alt~|A1| with two branches (but the approach generalises
to more branches).  The branches are defined using a definition such as
%
\begin{cspm}
branches =  <InPortBranch.c1, OutPortBranch.c2>
\end{cspm}
%
In different tests, we can vary whether the branches correspond to inports or
outports, so test different combinations.  We also create two instances |C1|
and |C2| of the channel module, corresponding to channels with
identities~|c1| and |c2|.  We combine these together in parallel with |A1|,
synchronising on the |begin| and |end| events that correspond to the alt
calling operations on a channel, or vice versa.

We combine these together with some threads.  One thread, which we denote
|AltThread|, repeatedly runs the alt.  The other threads, from set
|ChanThreads|, repeatedly call the main operations on the channels.

As an initial test, we can check whether this system is divergence-free.
However, recall that a thread waiting in the alt can perform a spurious
wakeup, denoted by the event |A1::spuriousWakeup|.  If we hide this event, it
turns out that the system can diverge, corresponding to a waiting thread
repeatedly having a spurious wakeup, rechecking the relevant condition, and
waiting again.  This is not a behaviour we should be concerned about: spurious
wakeups do happen, but they are rather rare; in practice, such spurious
wakeups will have a tiny effect on system performance.  If we keep the
spurious wakeups visible, then FDR verifies that the system cannot diverge: no
assertions fail, and the only possible source of infinite internal activity is
the spurious wakeups.

In the checks below, we hide the spurious wakeups.  The checks will be carried
out in the stable-failures model, so we should consider whether the potential
divergence is masking possible errors, by making critical states unstable.
But recall that we included in the model of the monitor a regulator process
that could block all spurious wakeups.  Thus for every state that is unstable
because of the possibility of a spurious wakeup, there is another, stable
state where the regulator blocks the spurious wakeup.  This way of abstracting
the spurious wakeups corresponds to Roscoe's \emph{lazy
  abstraction}~\cite{awr:UCS}. 

% Note: alternatively we could keep the spurious wakeups visible, and
% interleave with Chaos.  This makes no noticeable difference to the speed of
% the check.

We now consider the appropriate correctness condition.   This is an extension
of the correctness condition for a single channel from
Section~\ref{sec:syncchan}. 

We extend the |sync| events to include the identity of the channel being used:
|sync.t|$_1$|.t|$_2$|.c.x| represents a synchronisation between a sending
thread~|t|$_1$ and a receiving thread~|t|$_2$, both of which are
channel-threads, using channel~|c|, passing value~|x|.
%
We introduce similar events of the form |altSync.t|$_1$|.t|$_2$|.c.x| to
represent a synchronisation between a sending thread~|t|$_1$ and a receiving
thread~|t|$_2$, where one of the threads is the alt-thread.

We build lineariser processes for the channel-threads.  These are very similar
to as in Section~\ref{sec:syncchan}, so we elide some parts.  They are
extended for operations on either |C1| or |C2| (recall that a channel-thread
and alt-thread may use a port concurrently).  Further, they allow
synchronisations with the alt-thread.
%-- Lineariser for a channel thread.
\begin{cspm}
ChanThreadLin(me) = 
  C1::beginSend.me?x -> LinSending(me, x, c1, C1::endSend.me)
  [] C2::beginSend.me?x -> LinSending(me, x, c2, C2::endSend.me)
  [] ... -- similar for other operations 

LinSending(me, x, c, endChan) = 
  sync.me?other!c!x -> endChan.SendSuccess -> ChanThreadLin(me)
  [] altSync.me?altThread!c!x -> endChan.SendSuccess -> ChanThreadLin(me)
  [] isClosed.me.c -> endChan.Closed -> ChanThreadLin(me)

LinReceiving(me, c, endChan) = 
  sync?other!me!c?x -> endChan.ReceiveSuccess.x -> ChanThreadLin(me)
  [] altSync?altThread!me.c?x -> endChan.ReceiveSuccess.x -> ChanThreadLin(me)
  [] isClosed.me.c -> endChan.Closed -> ChanThreadLin(me)

LinSendingWithin(me, x, c, endChan) = 
  LinSending(me, x, c, endChan)
  [] timeout.me.c -> endChan.Timeout -> ChanThreadLin(me)

LinReceivingWithin(me, c, endChan) = 
  LinReceiving(me, c, endChan)
  [] timeout.me.c -> endChan.Timeout -> ChanThreadLin(me) 
\end{cspm}

We similarly build a lineariser process for the alt-thread.  We introduce an
event |allClosed| which will represent the linearisation point of an alt usage
that finds all the channels are closed and so throws an |AltAbort|.  
%
Let |inports| and |outports| be the sets of inports and outports that the alt
uses, and let the function |index| give the index of a particular branch.  The
definition below captures that before a successful return, the alt-thread must
perform a suitable synchronisation with a channel-thread, and that before
returning an |AltAbort|, it must detect that all the channels are closed. 
%
\begin{cspm}
AltLin(me) = 
  A1::beginAlt.me -> (
    altSync?other:ChanThreads!me?c:inPorts?x -> 
       A1::endAlt.me.AltReceive.index(InPortBranch.c).x -> AltLin(me)
    [] altSync.me?other:ChanThreads?c:outPorts?x -> 
          A1::endAlt.me.AltSend.index(OutPortBranch.c).x -> AltLin(me)
    [] allClosed -> A1::endAlt.me.AltAbort -> AltLin(me)
  )
\end{cspm}

We build a |ChanSpec| process for each channel.  Each extends the earlier
definition to allow |altSync| events, but only before the channel is closed.
Further, each can perform |allClosed| when the channel is closed; we
synchronise all the |ChanSpec| processes on this event, so it can happen only
when all channels are closed, as required.
%
\begin{cspm}
ChanSpec(c) = 
  sync?t1?t2!c?x -> ChanSpec(c) [] altSync?t1?t2!c?x -> ChanSpec(c) 
  [] timeout?t!c -> ChanSpec(c)   [] close?t!c -> ChanSpecClosed(c)
ChanSpecClosed(c) =
  isClosed?t!c -> ChanSpecClosed(c) [] allClosed -> ChanSpecClosed(c) [] close?t!c -> ChanSpecClosed(c)
\end{cspm}

We combine the different processes together, much as before.  We can then
verify that the system refines this specification in both the traces and
stable-failures models, showing that the system is synchronisation
linearisable and progressible.

\begin{window}[1,r,{
%
\vspace{0.5ex}
\begin{tabular}{\|crr\|}
Model & Time & States \\
D & 872s & 854M \\
F & 319s & 1.11B
%% With constants from Data
%% D & 920s & 834M \\ %???
%% F & 390s & 1.09B 
\end{tabular} % resp 2, 3 SF
},]
%
However, this approach suffers from a state-space explosion.  The table on the
right gives statistics about checks, for divergence freedom (D) and
progressibility (F); each test considers an alt with two branches (one inport
branch and one outport branch) and the corresponding two channels, used by
three threads (one alt-thread and two channel-threads).  The corresponding
tests with four threads were beyond the limits of the machine used. 
\end{window}
