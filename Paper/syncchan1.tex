\inlineCSP

\subsection{Synchronisation linearisation}
\label{sec:syncchan-analysis-1}

In this section, we describe, abstractly, the property that we expect to hold
of the basic model of the channel.  In the next section, we describe how to
capture this property using CSP and FDR.

Each successful execution of the |send| operation should synchronise with a
corresponding execution of |receive|, and vice versa: the two executions
should overlap in time, and the |receive| should return the value of the
|send|'s parameter.  

In~\cite{LL:synchronisation}, we introduced a correctness condition,
\emph{synchronisation linearisation}.  This property applies generally
to \emph{synchronisation objects}, i.e.~objects that allow two or more threads
to synchronise, and possibly to exchange data; examples include a synchronous
channel, the combination of an alt and associated channels, a barrier
synchronisation object (e.g.~\cite{andrews}), or an exchanger
(e.g.~\cite{Herlihy-Shavit}).  Below, we specialise the definition of
synchronisation linearisation to the case of a channel.  (The definition is
based upon that for
\emph{linearisation}~\cite{herlihy-wing}, the standard correctness condition
for concurrent datatypes.)  

A \emph{history} (or trace) of a channel is a sequence of |begin| and |end|
events, as in the previous subsection, corresponding to threads using the
channel.  It is convenient to label each event with an \emph{execution
identifier}, so that a |begin| and |end| event that correspond to the same
execution of an operation have the same execution identifier; for example
\begin{eqnarray*}
h & = & 
\trace{\CSPMM{beginSend}^1.t_1.3,\; \CSPMM{beginReceive}^2.t_2,\;
  \CSPMM{endReceive}^2.t_2.\CSPMM{ReceiveSuccess}.3,\; 
  \CSPMM{endSend}^1.t_1.\CSPMM{SendSuccess} }.
\end{eqnarray*}
We consider only valid histories, where each identifier appears on at most one
|begin| event and at most one |end| event, and where the identifier on each
|end| event matches that on an earlier |begin| event.  We say that a history
is \emph{complete} if for every |begin| event, there is a corresponding |end|
event, i.e.~there are no pending operation executions.

The idea is that each synchronisation should appear to take place between the
beginning and end of the two corresponding operation executions; different
synchronisations should occur in a one-at-a-time order.  We call the points at
which the synchronisations appear to take place \emph{synchronisation points}.
%
We use events of the form $\CSPMM{sync}^{i,j}.s.r.x$ to represent a
synchronisation between an execution of~|send| with execution identifier~$i$ by
thread~$s$, and an execution of~|receive| with execution identifier~$j$ by
thread~$r$, passing data value~$x$.
%
%\begin{definition}
A~\emph{synchronisation history} is a sequence of such \CSPM{sync} events.
Each execution identifier must appear at most once in the history. 
%\end{definition}
For example,
\begin{eqnarray*}
h_s & = & \trace{\CSPMM{sync}^{1,2}.t_1.t_2.3}
\end{eqnarray*}
is a synchronisation history that describes the (single) synchronisation in
the earlier channel history~$h$.

In the case of a basic synchronous channel, every such sequence of \CSPM{sync}
events is legal.  However, for some synchronisation objects, only certain
histories are legal.  For example, when we consider closing of channels, we
will require that all successful synchronisations precede the closing of the
channel.  In general, we identify a particular (prefix-closed) set of
histories as being \emph{legal}, based on the informal specification of the
synchronisation object.

The following definition describes when a history and synchronisation history
use the same execution identifiers.
\begin{definition}
We say that a complete history~$h$ of the channel, and a legal synchronisation
history~$h_s$ \emph{correspond} if:
%
\begin{itemize}
\item For every \CSPM{sync} event with identifier~$(i,j)$ in~$h_s$,\, $h$
  contains an execution of \CSPM{send} with identifier~$i$, and an execution
  of \CSPM{receive} with identifier~$j$;

\item For every execution of~\CSPM{send} with identifier~$i$ in~$h$,\, $h_s$
  contains a \CSPM{sync} event with identifier~$(i,j)$, for some~$j$;

\item For every execution of~\CSPM{receive} with identifier~$j$ in~$h$,\,
  $h_s$ contains a \CSPM{sync} event with identifier~$(i,j)$, for some~$i$.
\end{itemize}
\end{definition}
%
The earlier example histories~$h$ and~$h_s$ correspond.

%% The definition below specialises the definition from~\cite{LL:synchronisation}
%% to an SCL channel.
\begin{definition}
\label{def:sync-lin}
Let $h$ be a complete history of the channel, and $h_s$ a legal
synchronisation history, such that $h$ and~$h_s$ correspond.  We say that
$h_s$ is a \emph{synchronisation linearisation} of~$h$ if there is some way of
interleaving $h$ and~$h_s$ such that each $\CSPMM{sync}^{i,j}.s.r.x$ event is
between events $\CSPMM{beginSend}^i.s.x$ and
$\CSPMM{endSend}^i.s.\CSPMM{sendSuccess}$, and between events
$\CSPMM{beginReceive}^j.r$ and
$\CSPMM{endReceive}^j.r.\CSPMM{ReceiveSuccess}.x$.
\end{definition}

The earlier example synchronisation history~$h_s$ is a synchronisation
linearisation of the example~$h$: the event of~$h_s$ can be inserted after the
first two events of~$h$ to produce a suitable interleaving.

%% \begin{itemize}
%% \item Between each |beginSend.s.x| and |endSend.s.SendSuccess|, there is a
%%   unique event \CSPM{sync.s.r.x};

%% \item Between each |beginReceive.r| and |endReceive.r.ReceiveSuccess.x|, there
%%   is a unique event \CSPM{sync.s.r.x};

%% \item And vice versa, each |sync| event is between corresponding |beginSend|
%%   and |endSend| events, and also between corresponding |beginReceive| and
%%   |endReceive| events. 
%% \end{itemize}
%\end{definition}

The diagram below gives a larger example; it illustrates a particular history
of a channel and a synchronisation linearisation.
%
%\begin{figure}
\begin{center}
\unScalaMid
\begin{tikzpicture}[xscale = 0.9]
\draw[|-|] (0,0) -- node[above] {$\sm{send}^1.s_1\sm{.A}$} (4,0);
\draw[|-|] (4.5,0) -- node[above] {$\sm{send}^2.s_1\sm{.B}$} (8.5,0);
\draw[|-|] (0.5,-1) -- node[above] {$\sm{send}^3.s_2\sm{.A}$} (8,-1);
\draw[|-|] (1,-2) -- node[above] {$\sm{receive}^4.r.\sm{A}$} (3,-2);
\draw[|-|] (3.5,-2) -- node[above] {$\sm{receive}^5.r.\sm{B}$} (6.5,-2);
\draw[|-|] (7,-2) -- node[above] {$\sm{receive}^6.r.\sm{A}$} (9,-2);
\draw (1.9,0) \X; \draw (1.9,-2) \X; 
\draw[dashed] (1.9,0) -- (1.9,-2); % sync 1 and 3
\draw (1.9,-2.3) node {\small $\sm{sync}^{1,4}.s_1.r.\sm{A}$};
\draw (5.4,0) \X; \draw (5.4,-2) \X; 
\draw[dashed] (5.4,0) -- (5.4,-2); % sync 4 and 5
\draw (5.4,-2.3) node {\small $\sm{sync}^{2,5}.s_1.r.\sm{B}$};
\draw (7.6,-1) \X; \draw (7.6,-2) \X; 
\draw (7.6,-2.3) node {\small $\sm{sync}^{3,6}.s_2.r.\sm{A}$};
\draw[dashed] (7.6,-1) -- (7.6,-2); % sync 2 and 6 
\end{tikzpicture}
\scalaMid
\end{center}
%% \caption{Timeline representing the synchronisation example.}
%% \label{fig:sync-timeline}
%% \end{figure}
%
Time goes from left to right in the diagram.  Each horizontal line represents
an operation execution, with the end points representing the \CSPM{begin} and
\CSPM{end} events.  A corresponding synchronisation history is written at the
bottom, where each pair of ``$\cross$''s, linked by a dashed vertical line,
illustrates a possible synchronisation point of the corresponding executions,
and so defines the interleaving of the histories.

We considered only complete histories above.  We now generalise.  An
\emph{extension} of a (not necessarily complete) history~$h$ is formed by
adding zero or more |end| events corresponding to pending executions.  We
write $complete(h)$ for the subsequence of~$h$ formed by removing all |begin|
events of pending operation executions.
%
\begin{definition}
Let $h$ be a (not necessarily complete) history of the channel, and $h_s$ a
legal synchronisation history.  We say that $h_s$ is a \emph{synchronisation
  linearisation} of~$h$ if there is an extension~$h'$ of~$h$ such that $h_s$
is a synchronisation linearisation of $complete(h')$.  We say that $h$ is
synchronisation linearisable in this case.
%
We say that a synchronous channel is synchronisation linearisable if all of
its histories are synchronisation linearisable.
\end{definition}
%
Informally, the |end| events that are in~$h'$ but not in~$h$ correspond to
executions that have synchronised but not yet returned.

%% Thus a history (or trace) is synchronisation linearisable if it is possible
%% to identify synchronisation points with the desired properties.  This
%% corresponds to identifying which operation executions synchronise with one
%% another.

%% A more general definition is in~\cite{LL:synchronisation}.  In particular,
%% this includes the case where the synchronisation object maintains some state
%% between synchronisations, as we will require in
%% Section~\ref{sec:syncchan-closing}.

The definition is based on \emph{linearisation}~\cite{herlihy-wing}, the
standard correctness property for concurrent datatypes, where executions of
operations should appear to take place in a one-at-a-time order, each between
the beginning and end of that execution.  However, the two notions are
distinct: informally, synchronisation linearisation requires that operation
executions appear to take place in a two-at-a-time order.  More precisely,
synchronisation linearisation requires corresponding executions to overlap in
time.  A history where, say, a |send| returns before the corresponding
|receive| is invoked would not represent a correct synchronisation.  This
overlapping property cannot be captured directly with standard
linearisation~\cite{LL:synchronisation}.

%% The current model of the channel is (in the terminology
%% of~\cite{LL:synchronisation}) \emph{stateless}: no state is carried forward
%% from one synchronisation to another.  However, when we consider closing of the
%% channel, it will become \emph{stateful}, with two states, open or closed.
%% Other synchronisation objects have more interesting states.  \framebox{later?}

%%%%%%%%%%%%%%%%%%%%%%%%%%%%%%%%%%%%%%%%%%%%%%%%%%%%%%%%%%%%

\subsection{Model checking synchronisation linearisability}

We now describe how we use model checking to test synchronisation
linearisability of the channel.  (We drop the execution identifiers from
events: they were just convenient in defining synchronisation linearisation.)
We create an instance |C| of the channel module from
Figure~\ref{fig:basic-CSP-model}:  
\begin{cspm}
instance C = SyncChan(ThreadID, Data)
\end{cspm}
where |ThreadID| is a type of thread identifiers, and |Data| is a type of data
passed on the channel.
Below, expressions of the form |C::v|
represent a value~|v| from this instance.
%
We create a processes that represent threads (with identities
from a set |ThreadID|): each thread repeatedly calls the send and receive
operations,
% (represented by events on CSP channels |C::beginSend| and |C::beginReceive|),
and waits for them to return:
% (on CSP channels |C::endSend| and |C::endReceive|).
%
\begin{cspm}
Thread(me) = 
  C::beginSend.me?x -> C::endSend.me?res -> Thread(me)
  [] C::beginReceive.me -> C::endReceive.me?res -> Thread(me)
Threads = **||| t : ThreadID @ Thread(t)
\end{cspm}
%
We build a process |System| that combines the threads with the channel (the
function |runWithAndHide| from the channel module runs its argument in
parallel with the module's internal processes, and hides the internal events):
\begin{cspm}
System = C::runWithAndHide(Threads)
\end{cspm}

We now build a CSP specification process that allows precisely the traces that
are synchronisation linearisable.  As above, we use events of the form
\CSPM{sync.t}$\CSPcolour_1$\CSPM{.t}$\CSPcolour_2$\CSPM{.x} to represent a synchronisation point
between sender~$\CSPMM{t}\CSPcolour_1$ and receiver~$\CSPMM{t}\CSPcolour_2$, passing data
value~|x|.  We build a \emph{lineariser} process for each thread as follows,
which ensures that the |sync| events occur between the corresponding |begin|
and |end| events (the approach is based on that for standard
linearisation~\cite{gavin:lock-free-queue}).
%
\begin{cspm}
Lin(t) = 
  C::beginSend.t?x -> sync.t?other!x -> C::endSend.t.SendSuccess -> Lin(t)
  [] C::beginReceive.t -> sync?other!t?x -> C::endReceive.t.ReceiveSuccess.x -> Lin(t)
\end{cspm}
%
When sending, the thread can synchronise with any other thread~|other|,
passing its value~|x|.  When receiving, the thread can synchronise with any
other thread, accepting that thread's value~|x|; this |x| is subsequently
returned by the operation execution.

We combine the |Lin| processes in parallel, with their natural
alphabets, so the linearisers for threads $\CSPMM{t}_{\CSPcolour 1}$ and
$\CSPMM{t}_{\CSPcolour 2}$
synchronise on events of \CSPM{sync.t}$\CSPcolour_1$\CSPM{.t}$\CSPcolour_2$ and
\CSPM{sync.t}$\CSPcolour_2$\CSPM{.t}$\CSPcolour_1$.  
%
\begin{cspm}
alphaLin(t) =
  {|C::beginSend.t, C::endSend.t, C::beginReceive.t, C::endReceive.t|} £$\union$£
  {|sync.t.other, sync.other.t | other <- ThreadID-{t}|}
Spec£$_0$£ = **|| t <- ThreadID @ [alphaLin(t)] Lin(t)
\end{cspm} % £$\union$£
%
Thus each trace of |Spec|$\CSPcolour_0$ represents an interleaving of a history of the
channel with a synchronisation history, as required by
Definition~\ref{def:sync-lin}.  For example
\[
\begin{align}
\trace{\CSPMM{beginSend.T1.3},\; \CSPMM{beginReceive.T2},\;
  \CSPMM{sync.T1.T2.3}, \\
\qquad  \CSPMM{endReceive.T2.ReceiveSuccess.3}, \;
  \CSPMM{endSend.T1.SendSuccess}}
\end{align}
\]  
is a trace of~|Spec|$_0$, since the restriction to the events of |T1| is a
trace of~|Lin(T1)|, and the restriction to the events of |T2| is a
trace of~|Lin(T2)|.  On the other hand, the trace
\[
\trace{\CSPMM{beginSend.T1.3},\;  \CSPMM{sync.T1.T2.3}, 
  \CSPMM{endSend.T1.SendSuccess}}
\]
is not a trace of~|Spec|$_0$: the |sync| event would be blocked by |Lin(T2)|.

By hiding the synchronisation points, we obtain a process whose traces
represent precisely those histories that are synchronisation linearisable: 
\begin{cspm}
Spec = Spec£$_0$£ \ {|sync|}
\end{cspm}
For example, |Spec| would allow the correct trace
\[
\begin{align}
\trace{\CSPMM{beginSend.T1.3},\; \CSPMM{beginReceive.T2},\;
  \CSPMM{endReceive.T2.ReceiveSuccess.3}, \;
  \CSPMM{endSend.T1.SendSuccess}},
\end{align}
\]  
but not the incorrect trace 
\[
\trace{\CSPMM{beginSend.T1.3},\;   \CSPMM{endSend.T1.SendSuccess}}.
\]
We can thus use FDR to check that the system is synchronisation linearisable
as follows:
%
\begin{cspm}
assert Spec [T= System
\end{cspm}

This tests succeeds.  (Fuller results, for the complete model of the channel,
are given in Section~\ref{sec:syncchan-timed}.)

In particular, we can perform these tests with multiple threads, and so
include cases where multiple threads access the same port concurrently.  Hence
the analysis shows the channel works correctly when shared. 

%%%%%%%%%%%%%%%%%%%%%%%%%%%%%%%%%%%%%%%%%%%%%%%%%%%%%%%

\subsection{Synchronisation progressibility}

We now consider our progress property, \emph{synchronisation
  progressibility}~\cite{LL:synchronisation}.  This property assumes that the
system scheduler is fair, in the sense that if a thread is continuously
runnable, it will eventually be scheduled.  However, the assumption allows for
unfortunate scheduling, or for a thread to be repeatedly preempted, for
example when competing with other threads to acquire a lock.  Most modern
schedulers are fair in this sense.
The following definition captures our fairness assumption.
%
\begin{definition}
Given an execution, we say that an operation execution is \emph{pending} if it
has been called but not yet returned.  We say that an operation execution is
\emph{blocked} in a particular state if it is unable to perform a step, for
example because it is trying to acquire a lock held by another thread, or
waiting to receive a signal from another thread.

We say that an infinite execution is \emph{fair} if each pending operation
execution either (a)~eventually returns, (b)~is blocked in infinitely many
states, or (c)~performs infinitely many steps.  We consider a system execution
that reaches a deadlocked state (where every pending operation is permanently
blocked) to be infinite (it is infinite in time), and hence fair (under~(b)).
\end{definition}

%%%%%

Synchronisation progressibility requires that, under this fairness assumption,
if a synchronisation is possible, some such synchronisation will happen, and
the threads become able to return.  If several alternative synchronisations
are possible, then synchronisation progressibility allows any to happen.
Informally, threads do not get stuck if there are synchronisations that could
occur.  For example, if three threads call, respectively, |send(4)|, |send(5)|
and |receive| on the channel, then synchronisation progressibility requires
that the receiver synchronises with one of the senders, and those two threads
return.

%%%%%

The following definition captures a maximal sequence of synchronisations that
might occur, given a particular history of the channel.
%
\begin{definition}
Given a history~$h$ of a channel, and a legal synchronisation history~$h_s$,
we say that $h_s$ is a \emph{maximal synchronisation linearisation} of~$h$ if:
(a)~$h_s$ is a synchronisation linearisation of~$h$; and (b)~no proper
extension $h_s \cat \seq{e}$ is a synchronisation linearisation of~$h$.
\end{definition}
%
For example, the history 
\begin{eqnarray*}
h & = &\seq{\beginSend^1.s_1.4,\, \beginSend^2.s_2.5,\, \beginReceive^3.r }
\end{eqnarray*}
has two maximal synchronisation linearisations
\[
h_s^1  =   \seq{\sync^{1,3}.s_1.r.4}, \qquad\mbox{and}\qquad
h_s^2  =  \seq{\sync^{2,3}.s_2.r.5},
\]
corresponding to the two possible synchronisations.  Each describes one
possibility for all the synchronisations that might happen. 

%%%%%

The following definition captures the $\End$ events that we would expect to
happen given a particular sequence of synchronisations.
%
\begin{definition}
Given a history~$h$ of the synchronisation object and a maximal
synchronisation linearisation~$h_s$, we say that an $\End$ event is
\emph{anticipated} if it does not appear in~$h$, but the corresponding $\sync$
event appears in~$h_s$.
\end{definition}
%
For the above histories~$h$ and~$h_s^1$, the events
$\endSend^1.s_1.\SendSuccess$ and $\endReceive^3.r.\ReceiveSuccess.4$  are
anticipated: assuming $h_s^1$ describes the synchronisations that happen, we
would expect those $\End$ events to occur; if they do not, that is a failure
of progress.

%%%%%

\begin{definition}
Let $h$ be a history of a channel.  We say that $h$ is \emph{synchronisation
  progressible} if, from the state reached after~$h$, for every fair infinite
execution with no new $\Begin$ events, there is a maximal synchronisation
  linearisation~$h_s$ of~$h$ such that every anticipated $\End$ event~$e$
  eventually happens.

We say that a channel is synchronisation progressible if all of its histories
are synchronisation progressible.
\end{definition}

For the earlier history~$h$, synchronisation progressibility requires that
either $\endSend^1.s_1.\SendSuccess$ and $\endReceive^3.r.\ReceiveSuccess.4$
eventually happen (corresponding to~$h_s^1$), or $\endSend^2.s_2.\SendSuccess$
and $\endReceive^3.r.\ReceiveSuccess.5$ eventually happen (corresponding
to~$h_s^2$): one of the synchronisations should happen, and then the relevant
operations should return.

On the other hand, for the history $\trace{\beginSend^1.s.4}$, the only
maximal synchronisation linearisation is $\trace{}$, for which there are no
anticipated returns, and so synchronisation progressibility is vacuously
satisfied: it is fine for the |send| to get stuck in this case, since there is
no |receive| with which it could synchronise.

%%%%%%%%%%%%%%%%%%%%%%%%%%%%%%%%%%%%%%%%%%%%%%%%%%%%%%%

%% We now consider how to test for synchronisation progressibility using CSP and
%% FDR.  We need to check that anticipated $\End$ events are available, i.e.~not
%% refused.  To do this, we require that the system model is divergence-free,
%% since FDR models do not accurately capture refusal information in divergent
%% states.  However, as we will see, this is the case.   

We can test our CSP model for synchronisation progressibility by repeating the
earlier refinement test, except in the failures-divergences model (which
implies the refinement in the traces model):
%
\begin{cspm}
assert Spec [FD= System
\end{cspm}
%
Note that this test requires that the system cannot diverge, and so must
eventually reach a stable state.  The failures-divergences model records
refusal information only in stable states.  This reflects our assumption about
scheduling: unstable states are transient, because there must be some internal
step by a thread that is possible and that will be scheduled.  If a
synchronisation is possible, then the specification will reach a state where
relevant |end| events are available (i.e.~not refused).  The test, then,
requires that in stable states, the system also makes those |end| events
available.  If, in fact, there are two or more possible synchronisations, the
specification will choose nondeterministically which to perform, and so the
system may likewise perform any such synchronisations.

The property of \emph{lock freedom}~\cite{Herlihy-Shavit} requires that
eventually some thread returns, assuming the scheduler repeatedly schedules
threads.  Synchronisation progressibility is not the same as lock freedom,
because lock freedom does not assume that the scheduler is fair.  Indeed, SCL
channels are not lock-free, because of the use of a lock: if one thread
obtains the lock, and then the scheduler only schedules other threads, those
other threads would be blocked trying to obtain the lock, and so no thread
would return.  However, under our assumption of fair scheduling, the thread
holding the lock would eventually be scheduled and so release the lock.

The property of \emph{wait freedom} requires that if a particular thread is
repeatedly scheduled, it eventually returns.  The property of
\emph{obstruction freedom} requires that if a thread executes in isolation
(i.e.~no other thread is scheduled), then it eventually returns.  SCL channels
are neither wait-free not obstruction-free, for the same reason as above.
Hence synchronisation progressibility is also distinct from these two
properties.

%%%%%%%%%%%%%%%%%%%%%%%%%%%%%%%%%%%%%%%%%%%%%%%%%%%%%%%

%% We can also test the same refinement in the stable failures model (which
%% implies the refinement in the traces model).
%% %
%% \begin{cspm}
%% assert Spec [F= System
%% \end{cspm}
%% %
%% This captures a useful progress property, which says that if a synchronisation
%% is possible (i.e.~there is at least one |send| and one |receive| operation
%% that have been called but not yet returned), then some such synchronisation
%% must happen (since an internal |sync| event is available), and so the relevant
%% threads reach a state where they can return.  (If several different
%% synchronisations are possible, then the refinement test allows any to happen.)
%% Informally, threads don't get stuck unnecessarily.  We call this property
%% \emph{synchronisation progressiblity}~\cite{LL:synchronisation}.  This
%% property corresponds to an assumption that the scheduler is fair in the sense
%% that if a thread is continuously runnable, it will eventually be scheduled and
%% so able to make progress; however, this doesn't prevent the thread being
%% repeatedly preempted, for example when competing with other threads to acquire
%% a lock. 

Finally, note that the fact that |System| does not diverge verifies that no
thread in the implementation throws an exception. 

%% Finally, as discussed above, we can test that |System| does not diverge, to
%% verify that no thread in the implementation throws an exception.  Further,
%% this verifies that threads cannot perform an infinite amount of internal
%% activity without any thread returning.  This test can be combined with the
%% previous by testing for refinement in the failures-divergences model.
%% %
%% \begin{cspm}
%% assert Spec [FD= System
%% \end{cspm}

It turns out that our specification is equivalent to that of Welch and
Martin~\cite{welch-martin} when restricted to non-shared channels (which they
consider).  However, we consider our specification to be an instance of a more
general pattern, for synchronisation objects, rather than a single-purpose
specification.
