\begin{abstract}
\framebox{\ldots}
\end{abstract}

\section{Introduction}

Scala Concurrency Library (SCL) is a library of concurrency primitives for the
Scala programming language.  It was developed for teaching concurrent
programming to students, and includes support for message-passing concurrency,
monitors and semaphores.  In this paper we analyse the message-passing
primitives: we build CSP models of them, and then use the model checker FDR to
test for correctness.  To our surprise, the analysis revealed a bug in the
implementation.

We start by describing relevant aspects of SCL\@.  Program threads can send
and receive messages using \emph{channels}.  If |c| is a channel, then the
command |c!x| sends the value~|x| on~|c|; the expression~|c?()| receives and
returns a value.  We consider just \emph{synchronous} channels in this paper:
the sending thread much wait until there is a thread willing to receive.
Channels are typed: the type |SyncChan[A]| represents synchronous channels
that send data of type~|A|.  A channel is composed of an \emph{outport}, where
values are sent, and an \emph{inport}, where values are received.

Channels also have timed operations.  The operation |sendWithin(delay)(x)|
attempts to send |x|, but if it is unable to synchronise with a receiving
thread within |delay|\,ms, it times out; it returns a boolean to indicate
whether sending was successful.  Similarly, the operation
|receiveWithin(delay)| attempts to receive, but if it is unable to synchronise
with a sending thread within |delay|\,ms, it times out; it returns a value
|Some(x)| to indicate that it successfully received~|x|, or |None| to indicate
that it timed out.

Ports may be shared: multiple threads may try to send or receive on the same
port concurrently; the channel is responsible for pairing off a sender with a
receiver. 

A channel can be closed (by the |close| operation): subsequently, an attempt
to send or receive on the channel will throw a |Closed| exception.

An \emph{alt} allows a thread to communicate on one of several channels,
whichever is available for communication first.  The following example
illustrates the usage.
%
\begin{scala}
alt(
  in =?=> { x => println(x) }
  | out =!=> { 42 } ==> { println("42 sent") }
)
\end{scala}
%
An alt consists of a number of branches, separated by ``\SCALA{\|}''.  An
\emph{inport branch} is denoted ``\SCALA{in =?=> f}'' where |in| is a channel
(or inport), and |f| is a function that takes an argument of the type of~|in|;
we call~|f| a \emph{continuation} (above, ``\SCALA{x => println(x)}'' denotes
the function that takes argument~|x| and executes~|println(x)|).  An
\emph{outport branch} is denoted ``\SCALA{out =!=> e ==> cont}'', where |out|
is a channel (or outport), |e| is an expression which evaluates to a value of
the type of~|out|, and |c| is a computation, which we again call a
continuation (this continuation is optional).  The alt waits until there is
another thread ready to communicate at the other end of the channel
corresponding to one of the branches.  In the case of an inport branch, the
continuation is applied to the value received.  In the case of an outport
branch, the value of the expression is sent, and the continuation (if present)
is executed.

A branch may have a boolean \emph{guard}: a branch may be selected only if the
guard evaluates to true (however, we do not model guards in our analysis).  As
an example, the following code implements a bounded buffer, with maximum
capacity~|Bound|.
%
\begin{scala}
val queue = new Queue[Int]
while(true){
  alt(
    queue.length < Bound & in =?=> { x => q.enqueue(x) }
    | queue.nonEmpty & out =!=> { q.dequeue() }
  )    
}
\end{scala}
%
We say that a branch is \emph{feasible} if the port has not been closed and
the guard is true.  Above, the inport branch is feasible only if the buffer is
not full; the outport branch is feasible only if the buffer is not empty.  If
no branch of an alt is feasible, it throws an |AltAbort| exception. 

There are two restrictions on the usage of alts: a port may not be
simultaneously feasible in two alts (although a non-alt thread may use a port
concurrently to an alt); and both ports of a channel may not simultaneously be
feasible in alts.  The implementation throws an exception if these
restrictions are not respected. 

In this paper, we build CSP models of the implementations of channels and
alts.  We then use the model checker FDR to analyse them against appropriate
specifications. 
%
The implementations of channels and alts are tricky: each has multiple modes
of operation, and can be used concurrently by multiple threads.  These factors
also provide a challenge to the analysis.
%
We do not include the Scala implementation here, because there is so much
code; but it can be obtained from~\framebox{???}.  Likewise, we do not include
the CSP model of the implementation, but instead concentrate on the
specification, which we consider more interesting; all the CSP can be obtained
from~\framebox{???}.

The rest of the paper is structured as follows.  In Section~\ref{sec:csp} we
give a brief overview of the syntax and semantics of CSP\@.  In
Section~\ref{sec:syncchan} we consider synchronous channels.  We describe
different aspects of channels incrementally, in the interests of clarity.  We
start by considering just the (untimed) send and receive operations: we give
an overview of the implementation, and of the corresponding part of the CSP
model.  We then describe the correctness condition for these operations,
namely \emph{synchronisation linearisation}~\cite{LL:synchronisation}.  We
also describe a related progress property, \emph{synchronisation
  progressibility}.  We present the corresponding CSP specification and
refinement check for each property.  extend our analysis to consider the
closing of channels: this is of particular interest, because the analysis
revealed an error in an earlier implementation.  Fixing this error, required
fairly substantial changes to the implementation.  Performing this analysis
helps to clarify what the correctness condition should be, and so helps to
focus on the critical point.  We then extend the analysis to the timed send
and receive operations (but ignoring the interactions with alts at this
point).

In Section~\ref{sec:alt} we consider alts.  We describe the high-level design,
concerning the interactions (via operation calls) between alts and channels,
sketch some implementation details, and describe aspects of the CSP model.  We
then describe a direct analysis: we consider a system constructed from an alt
with a fixed number of branches, and associated channels; we construct a
corresponding CSP specification for synchronisation linearisability and
progressibility.  This analysis was useful in helping to develop a correct
implementation: it revealed various flaws with earlier versions.  However, the
analysis suffers from a state-space explosion, and so it's possible to analyse
only rather small systems.

In Section~\ref{sec:compositional}, we perform an alternative, compositional,
verification.  We build a more abstract CSP description of a synchronous
channel, describing the way it reacts to operation calls and interacts with
alts, but abstracting away from details of the implementation: we call this an
\emph{idealised channel}.  We show that the CSP model of the channel
implementation refines this idealised channel.  Likewise, we build an
idealised model of an alt, and show that it is refined by the model of the
implementation.  Finally, we combine the idealised alt with a fixed number of
idealised channels, use FDR to analyse the combination, and argue that this
implies correctness for the corresponding combination of the implementation
models.  This approach scales much better than the direct analysis.  A
challenge of this approach is that the implementations of a channel and an alt
each assumes that other components act correctly, i.e.~follows the protocol
that defines interactions between them: we describe how to deal with this
challenge.

\framebox{\ldots}

We employed various techniques in our CSP modelling.  We present some of these
in an appendix: they are rather orthogonal to the main focus of this paper,
but we believe they would be useful elsewhere.

We consider our main contributions to be the following:
%
\begin{itemize}
\item The modelling of a fairly large body of code, and the development of
  related modelling techniques;

\item The specification of synchronisation linearisation and progressibility,
  and the illustration of how they can be tested using model checking; 

\item The demonstration that this technique can discover real bugs on code;

\item The demonstration of compositional verification, in particular where
  each component makes assumptions about the correct behaviour of other
  components. 
\end{itemize}


%%%%%%%%%%%%%%%%%%%%%%%%%%%%%%%%%%%%%%%%%%%%%%%%%%%%%%%

\subsection{TO DO}

Alt sent values are calculated only when sent.  
