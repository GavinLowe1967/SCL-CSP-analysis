\begin{abstract}
We carry out an analysis of message-passing concurrency primitives, namely a
synchronous channel and an alt (alternation) construct.  We model these
primitives using the process algebra CSP, and analyse them using the model
checker FDR.  We consider the correctness conditions of synchronisation
linearisation and progressibility: we show how these can be captured in CSP\@.
Our analysis discovered an error in a previous implementation.  A direct
analysis of the composition of an alt and corresponding channels scales quite
poorly.  To overcome this, we perform a compositional analysis: we show that a
channel and an alt each satisfy a more abstract description; and show that the
composition of these abstract descriptions satisfies synchronisation
linearisation and progressibility.
\end{abstract}

%%%%%%%%%%%%%%%%%%%%%%%%%%%%%%%%%%%%%%%%%%%%%%%%%%%%%%%

\section{Introduction}

Scala Concurrency Library (SCL) is a library of concurrency primitives for the
Scala programming language.  It was developed for teaching concurrent
programming to students, and includes support for message-passing concurrency,
monitors and semaphores.  In this paper we analyse the message-passing
primitives: we build CSP models of them, and then use the model checker FDR to
test for correctness.  To our surprise, the analysis revealed a bug in the
implementation.

We start by describing relevant aspects of SCL\@.  Program threads can send
and receive messages using \emph{channels}.  If |c| is a channel, then the
command |c.send(x)| (also written as |c!x|) sends the value~|x| on~|c|; the
expression |c.receive()| (also written as~|c?()|) receives and
returns a value.  We consider just \emph{synchronous} channels in this paper:
the sending thread much wait until there is a thread willing to receive, so
that the two invocations synchronise.  Channels are typed: the type
|SyncChan[A]| represents synchronous channels that send data of type~|A|.  A
channel is composed of an \emph{outport}, where values are sent, and an
\emph{inport}, where values are received.

Channels also have timed operations.  The operation |sendWithin(delay)(x)|
attempts to send |x|, but if it is unable to synchronise with a receiving
thread within |delay|\,ms, it times out; it returns a boolean to indicate
whether sending was successful.  Similarly, the operation
|receiveWithin(delay)| attempts to receive, but if it is unable to synchronise
with a sending thread within |delay|\,ms, it times out; it returns a value
|Some(x)| to indicate that it successfully received~|x|, or |None| to indicate
that it timed out.

Ports may be shared: multiple threads may try to send or receive on the same
port concurrently; the channel is responsible for pairing off a sender with a
receiver. 

A channel can be closed (by the |close| operation): subsequently, an attempt
to send or receive on the channel will throw a |Closed| exception.

An alternation, or \emph{alt}, allows a thread to communicate on one of
several channels, whichever is available for communication first.  The
following example illustrates the usage.
%
\begin{scala}
alt(
  in =?=> { x => println(x) }
  | out =!=> { 42 } ==> { println("42 sent") }
)
\end{scala}
%
An alt consists of a number of branches, separated by ``\SCALA{\|}''.  An
\emph{inport branch} is denoted ``\SCALA{in =?=> f}'' where |in| is a channel
(or inport), and |f| is a function whose argument matches the type of~|in|; we
call~|f| a \emph{continuation} (above, ``\SCALA{x => println(x)}'' denotes the
function that takes argument~|x| and executes~|println(x)|).  An \emph{outport
  branch} is denoted ``\SCALA{out =!=> e ==> cont}'', where |out| is a channel
(or outport), |e| is an expression whose value matches the type of~|out|, and
|c| is a computation, which we again call a continuation (this continuation is
optional).  The alt waits until there is another thread ready to communicate
at the other end of the channel corresponding to one of the branches, at which
point the two threads can synchronise to transmit a value.  In the case of an
inport branch, the continuation is applied to the value received.  In the case
of an outport branch, the value of the expression is sent, and the
continuation (if present) is executed.

A branch may have a boolean \emph{guard}: a branch can be selected only if the
guard evaluates to true.  As an example, the following code implements a
bounded buffer, with maximum capacity~|Bound|.
%
\begin{scala}
val queue = new scala.collection.mutable.Queue[Int]
while(true){
  alt(
    queue.length < Bound & in =?=> { x => q.enqueue(x) }
    | queue.nonEmpty & out =!=> { q.dequeue() }
  )    
}
\end{scala}
%
This example also illustrates that the calculation of the value sent in an
outport branch might have side effects; therefore the relevant expression is
evaluated only once the alt commits to communication via that branch.

We say that a branch is \emph{feasible} if the port has not been closed and
the guard is true.  Above, the inport branch is feasible only if the buffer is
not full; the outport branch is feasible only if the buffer is not empty.  If
no branch of an alt is feasible, it throws an |AltAbort| exception. 

The construct |serve(branches)| is like an |alt(branches)| that is executed
repeatedly, except any |AltAbort| exception is caught.  Hence it runs
repeatedly until no branch is feasible. 

There are two restrictions on the usage of alts: a port may not be
simultaneously feasible in two alts (although a port may be simultaneously
feasible in an alt and used by a non-alt thread); and both ports of a channel
may not simultaneously be feasible in alts.  The implementation throws an
exception if these restrictions are not respected.

In this paper, we build CSP models of the implementations of channels and
alts.  We then use the model checker FDR to analyse them against appropriate
specifications. 
%
The implementations of channels and alts are tricky: each has multiple modes
of operation, and can be used concurrently by multiple threads.  These factors
also provide a challenge to the analysis.
%
We do not include the Scala implementation here, because there is so much
code; but it can be obtained via the paper's web page~\framebox{???}.
Likewise, we do not include the CSP model of the implementation, but instead
concentrate on the specification, which we consider more interesting; all the
CSP can be obtained from the paper's web page.

The rest of the paper is structured as follows.  In Section~\ref{sec:csp} we
give a brief overview of the syntax and semantics of CSP\@.  In
Section~\ref{sec:syncchan} we consider synchronous channels.  We describe
different aspects of channels incrementally, in the interests of clarity.  We
start by considering just the (untimed) send and receive operations: we give
an overview of the implementation, and of the corresponding part of the CSP
model.  We then describe the correctness condition for these operations,
namely \emph{synchronisation linearisation}~\cite{LL:synchronisation}.  We
also describe a related progress property, \emph{synchronisation
  progressibility}.  We present the corresponding CSP specification and
refinement check for each property.  We then extend our analysis to consider
the closing of channels: this is of particular interest, because the analysis
revealed an error in an earlier implementation.  Fixing this error, required
fairly substantial changes to the implementation.  Performing the analysis in
this paper helped to clarify what the correctness condition should be, and so
helped to focus on the critical point.  We then extend the analysis to the
timed send and receive operations (but ignoring the interactions with alts at
this point).

In Section~\ref{sec:alt} we consider alts.  We describe the high-level design
in terms of the interactions (via operation calls) between alts and channels;
we  sketch some implementation details, and describe aspects of the CSP
model.  We then describe a direct analysis: we consider a system constructed
from an alt with a fixed number of branches, and associated channels; we
construct a corresponding CSP specification for synchronisation
linearisability and progressibility.  This analysis was useful in helping to
develop a correct implementation: it revealed various flaws with earlier
versions.  However, the analysis suffers from a state-space explosion, and so
it's possible to analyse only rather small systems.

In Section~\ref{sec:compositional}, we perform an alternative, compositional,
verification.  We build a more abstract CSP description of a synchronous
channel, describing the way it reacts to operation calls and interacts with
alts, but abstracting away from details of the implementation: we call this an
\emph{idealised channel}.  We show that the CSP model of the channel
implementation refines this idealised channel.  Likewise, we build an
idealised model of an alt, and show that it is refined by the model of the
implementation.  A challenge for this analysis is that each component makes
assumptions about other components with which it is composed: we describe how
we capture these assumptions.  Finally, we combine the idealised alt with a
fixed number of idealised channels, use FDR to analyse the combination, and
argue that this implies correctness for the corresponding combination of the
implementation models.  This approach scales much better than the direct
analysis.  A challenge of this approach is that the implementations of a
channel and an alt each assumes that other components act correctly,
i.e.~follow the protocol that defines interactions between them: we describe
how to deal with this challenge.

We sum up in Section~\ref{sec:conc}.
%
We employed various techniques in our CSP modelling.  We present some of these
in an appendix: they are rather orthogonal to the main focus of this paper,
but we believe they would be useful elsewhere.

We consider our main contributions to be the following:
%
\begin{itemize}
\item The modelling of a fairly large body of code, larger than previous
  similar analyses;

\item The development of related modelling techniques;

\item The illustration of how synchronisation linearisation and
  progressibility can be tested using model checking;

\item The demonstration that this technique can discover real bugs on code;

\item The demonstration of compositional verification, in particular where
  each component makes assumptions about the correct behaviour of other
  components. 
\end{itemize}


%%%%%%%%%%%%%%%%%%%%%%%%%%%%%%%%%%%%%%%%%%%%%%%%%%%%%%%

\subsection{Related work}

CSP has been used to analyse message passing concurrency primitives on a
number of previous occasions.
%
Welch and Martin~\cite{welch-martin} present a model of Java multi-threading
(in particular, monitors), including a model of a channel within their own
concurrency API\@.
%
I~\cite{gavin:alt} use CSP to derive and verify a generalisation of the
alt construct.  
%
In~\cite{gavin:OneOne}, I used CSP to discover the cause of a deadlock in an
implementation of a synchronous channel. 

CSP has also been used more widely to analyse concurrent systems. 
%
Lawrence~\cite{lawrence} describes the use of CSP and FDR in an
industrial setting, for the analysis of a system for connection pooling.
% 
Mota and Sampaio~\cite{mota+sampaio} analyse a subset of the
control system of a satellite, modelled in CSP-Z\@.  
%
CSP and FDR have been widely used to analyse security protocols
(e.g.~\cite{gavin:NSFDR}).

Hopkins and Roscoe~\cite{hopkins-roscoe}, and Pay~\cite{alex:project} describe
compilers for compiling from a simple shared-variable language into CSP\@.



%% \subsection{TO DO}

%% Alt sent values are calculated only when sent.  Actually, this is obvious
%% from the point of view of an alt.
%% %
%% \begin{itemize}
%% \item The value is evaluated during registration iff that registration
%%   succeeds; and in that case, the appropriate value is used by the alt;

%% \item The value is evaluated during maybeSend iff that call to maybeSend
%%   succeeds; and  in that case, the appropriate value is used by the alt.
%% \end{itemize}
%% %
%% Then we use the relationship between alt sends and channel receives to deduce
%% the result. 

