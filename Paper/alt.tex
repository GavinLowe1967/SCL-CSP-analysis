\section{Alts}
\label{sec:alt}

\inlineScala

We now consider alts.  We start by describing the high-level interactions
between an alt and channels, and how those interactions are implemented within
alts and channels.  We then outline how these interactions are modelled in
CSP, and describe a direct analysis of an alt and associated channels.

%%%%%%%%%%%%%%%%%%%%%%%%%%%%%%%%%%%%%%%%%%%%%%%%%%%%%%%

\subsection{High-level design}

We describe the interactions between an alt and relevant channels via function
calls.  We call this the \emph{alt protocol}. 

When an alt runs, it starts by registering, in turn, with each of the relevant
ports whose guard evaluates to |true|.  This asks the port whether it is ready
to communicate.  The alt calls a function
%
\begin{scala}
def registerIn(alt: AltT, index: Int, iter: Int): RegisterInResult[A]  
\end{scala}
%
on each of its inports, where |alt| is a reference to the calling alt, |index|
is the index of the branch within the alt, and |iter| is an iteration number
(used only for assertions).  The function returns a result of type
|RegisterInResult[A]|, where |A| is the type of data passed by the port, of
one of the following forms.
%
\begin{description}
\item[\rm{\scalastyle RegisterInSuccess(x)}:] the port was willing to
  communicate, and the alt has received~|x| from it;

\item[\rm{\scalastyle RegisterInWaiting}:] the port is not currently willing to
  communicate (but the registration has been recorded); 

\item[\rm{\scalastyle RegisterInClosed}:] the port has been closed.
\end{description}
%
Similarly, the alt calls a function
%
\begin{scala}
def registerOut(alt: AltT, index: Int, iter: Int, value: () => A): RegisterOutResult
\end{scala}
on each of its outports, where |alt|, |index| and |iter| are as for
|registerIn|, and |value| is a computation that, when evaluated, produces the
value to be sent.  The function returns a result of one of the following
forms.
%
\begin{description}
\item[\rm{\scalastyle RegisterOutSuccess}:] the port was willing to
  communicate, and the alt has sent it a value;

\item[\rm{\scalastyle RegisterOutWaiting}:] the port is not currently willing to
  communicate (but the registration has been recorded); 

\item[\rm{\scalastyle RegisterOutClosed}:] the port has been closed.
\end{description}

If one of the registrations is successful, the alt deregisters from the
waiting branches, via functions |deregisterIn| and |deregisterOut|.  It then
executes the continuation of the successful branch.  

\def\IP{\emph{IP}}
\def\OP{\emph{OP}}

Figure~\ref{fig:alt-1}
gives an example: an alt first registers unsuccessfully with an inport~\IP;
then it registers successfully from an outport~\OP; and finally deregisters
from~\IP. 

%%%%%

\begin{figure}
\begin{center}
%\unScalaMid
\def\height{5.5} % height of sequence diagram
\def\gap{3.5} % gap between columns
\begin{tikzpicture}[yscale = 0.8, >= angle 60]
\draw (0,0) node(alt){Alt};
\draw (alt) -- ++ (0,-\height);
\draw (alt)++(\gap,0) node(c1){\IP};
\draw (c1) -- ++(0,-\height);
\draw (c1)++(\gap,0) node(c2){\OP};
\draw (c2) -- ++(0,-\height);
% Register with IP
\draw[->] (alt)++(0,-1) -- node[above]{\scalashape\small registerIn}
  ++ (\gap,0);
\draw[<-] (alt)++(0,-2) -- node[above]{\scalashape\small RegisterInWaiting} 
  ++ (\gap,0);
% Register with OP
\draw[->] (alt)++(0,-3) -- node[above, near end]{\scalashape\small registerOut}
  ++ (2*\gap,0);
\draw[<-] (alt)++(0,-4) -- 
  node[above, near end]{\scalashape\small RegisterOutSuccess}
  ++ (2*\gap,0);
% Deregister with IP
\draw[->] (alt)++(0,-5) -- node[above]{\scalashape\small deregisterIn}
  ++ (\gap,0);
\end{tikzpicture}
\end{center}
\caption{Sequence diagram illustrating an alt that registers successfully with
  a port. \label{fig:alt-1}}
\end{figure}

%%%%%

If no registration is successful, and the alt finds that every port is closed
or has a guard that is false, then it throws an |AltAbort| exception.
Otherwise, it waits for a callback from one of the ports with which it is
registered, of one of the following forms.
%
\begin{itemize}
\item If the alt is registered at the inport of a channel, and another thread
  tries to send on the channel, it calls
\begin{scala}
def maybeReceive(value: A, index: Int, iter: Int): Boolean
\end{scala}
%
where |value| is the value it is trying to send to the alt, and |index| and
|iter| match the values provided during registration.  This asks the alt
whether it is still willing to receive from the inport.

\item If the alt is registered at the outport of a channel, and another thread
  tries to receive on the channel, it calls
%
\begin{scala}
def maybeSend[A](index: Int, iter: Int): Option[A]
\end{scala}
%
This asks the alt whether it is still willing to send a value to the inport.

\item If another thread closes the channel, it calls 
%
\begin{scala}
def portClosed(index: Int, iter: Int)
\end{scala}
\end{itemize}

If the alt receives a call of |maybeReceive| or |maybeSend|, it responds
positively to the first such call, returning |true| to a |maybeReceive|, or
|Some(x)| to a |maybeSend|, where~|x| is the value it sends.  It then
deregisters from the remaining branches.  
%
Finally, the alt
executes the continuation of the successful branch.
%
Figure~\ref{fig:alt-2} gives an example of a successful callback of
|maybeSend| from an outport with which the alt is registered.

\begin{figure}
\begin{center}
%\unScalaMid
\def\height{7.5} % height of sequence diagram
\def\gap{3.5} % gap between columns
\begin{tikzpicture}[yscale = 0.8, >= angle 60]
\draw (0,0) node(alt){Alt};
\draw (alt) -- ++ (0,-\height);
\draw (alt)++(\gap,0) node(c1){\IP};
\draw (c1) -- ++(0,-\height);
\draw (c1)++(\gap,0) node(c2){\OP};
\draw (c2) -- ++(0,-\height);
% Register with IP
\draw[->] (alt)++(0,-1) -- node[above]{\scalashape\small registerIn}
  ++ (\gap,0);
\draw[<-] (alt)++(0,-2) -- node[above]{\scalashape\small RegisterInWaiting} 
  ++ (\gap,0);
% Register with OP
\draw[->] (alt)++(0,-3) -- node[above, near end]{\scalashape\small registerOut}
  ++ (2*\gap,0);
\draw[<-] (alt)++(0,-4) -- 
  node[above, near end]{\scalashape\small RegisterOutWaiting}
  ++ (2*\gap,0);
% Call back
\draw[<-] (alt)++(0,-5) -- node[above, near end]{\scalashape\small maybeSend}
  ++ (2*\gap,0);
\draw[->] (alt)++(0,-6) -- node[above, near end]{\scalashape\small Some(x)}
  ++ (2*\gap,0);
% Deregister with IP
\draw[->] (alt)++(0,-7) -- node[above]{\scalashape\small deregisterIn}
  ++ (\gap,0);
\end{tikzpicture}
\end{center}
\caption{Sequence diagram illustrating a successful callback from a
  port.  \label{fig:alt-2}}
\end{figure}

%%%%%

If the alt receives multiple callbacks to |maybeReceive| or |maybeSend|
(including during deregistration), it responds negatively, returning |false|
or |None|, respectively.
%
If all the channels with which the alt is registered call |portClosed|, the
alt throws an |AltAbort|. 

%%%%%%%%%%%%%%%%%%%%%%%%%%%%%%%%%%%%%%%%%%%%%%%%%%%%%%%%%%%%

\subsection{Implementation details}

We now describe some details of the implementation.  Below we will use the
term ``alt-thread'' for the thread that is running the alt, and
``channel-thread'' for a thread performing an operation in a channel that
makes a callback to the alt. 

Each call of a function on a channel (registering or deregistering) uses that
channel's lock, to avoid races.  

The implementation of the alt is based on a monitor, more specifically a
monitor provided by the Java Virtual Machine (JVM).  (JVM monitors are more
efficient than SCL monitors, because they are implemented directly in the JVM;
however, they do not allow targeting of signals.)  

The alt-thread holds the alt's lock throughout the registration phase.  Each
callback function has to obtain this lock, so those functions are blocked
until registration is complete. 

However, it would be a mistake for the alt-thread to continue to hold the
alt's lock during deregistration, for this could lead to deadlocks.  Suppose
it did continue to hold the lock, and consider the case that the alt-thread is
trying to deregister from channel~$c$, at the same time that a channel-thread
is trying to perform a callback from~$c$: the channel-thread holds the lock
on~$c$, so the deregistration would be blocked; but the alt-thread holds the
lock on the alt, so the callback would be blocked.  The alt-thread therefore
releases the lock during deregistration. 

The implementation uses a variable |done| that is set to |true| when a
branch is found that is willing to communicate, either during registration or
as the result of a callback.  If a callback of |maybeSend| or |maybeReceive|
finds that |done| is |true|, it can return a negative result.  Otherwise, it
stores relevant information (like the value it is sending in the case of
|maybeReceive|, and the index of the relevant branch), evaluates the value to
be received in the case of |maybeSend|, sets |done| to |true|, signals to the
waiting alt-thread, and returns a positive result.

If the alt-thread fails to communicate during registration, and not all
ports are closed, it waits to receive a signal.  If |done| is true, it
deregisters from other branches and runs the continuation of the relevant
branch.  Otherwise, if all ports are closed, it throws an |AltAbort|. 

%% Why not hold lock during registration??

We now describe how the implementation of channels is extended to deal with
alts.

The |registerIn| function checks whether another alt is currently registered
(which would be an error), and if so throws an exception.  If the channel is
closed, it returns |RegisterInClosed|.  If there is a waiting sender
(corresponding to the variable |status| holding |Filled|), it acts like a
standard receive, setting status to |Read| and signalling to the sender, and
then returns a |RegisterInSuccess| result.  Otherwise it records the
registration, and returns |RegisterInWaiting|.  

The |registerOut| function is somewhat similar.  If there are waiting
receivers, it first waits for any current exchange to finish (i.e.~for
|status| to hold |Empty|).  If there are still waiting receivers, it continues
as for a standard send: it stores its value, sets |status| to |Filled| and
signals to a receiver.  It then waits on the |continue| condition for a
receiver to take the value, and resets |status| to |Empty|.  This latter wait
is necessary to ensure correct synchronisation: without it, the value could be
taken by a new receiver that calls |receive| only after the alt-thread has
returned.  

The deregister functions simply clear the registration information.

If a call of |send| or |sendWithin| finds that there is an alt registered at
the inport, it calls |maybeReceive| on that alt, and reacts accordingly.
Calls to |receive| or |receiveWithin| act similarly, calling |maybeSend|.  

Finally, if a channel is closed, it calls |portClosed| on any registered alt. 

%%%%%%%%%%%%%%%%%%%%%%%%%%%%%%%%%%%%%%%%%%%%%%%%%%%%%%%%%%%%

\subsection{CSP modelling for alts}

\inlineCSP

We now describe how to model an alt, and its interactions with channels, using
CSP\@.  Much of the construction of the model follows a similar form to the
model of a channel.  We highlight the main differences. 

A JVM monitor is modelled by a CSP module, in a similar way to an SCL monitor.
A process records which thread, if any, currently holds the lock, and which
threads are currently waiting for a signal.  One difference, however, concerns
a bug in the implementation of JVM monitors: a thread that is waiting for a
signal may resume, even though it has received no signal!  This is known as a
\emph{spurious wakeup}.  The implementation of an alt guards against spurious
wakeups by performing suitable checks when resuming after a wait, and waiting
again if appropriate.

Our CSP model of a monitor reflects the possibility of spurious wakeups,
allowing a waiting thread~|t| to resume either as the result of a signal, or a
spurious wakeup, modelled by the event |spuriousWakeup.t|.  We run this
monitor process in parallel with a \emph{regulator process} that
nondeterministically chooses whether or not to allow a spurious wakeup.  This
last point is important: if we allowed unrestricted spurious wakeups, there is
a danger that our analysis would seem to show that suitable progress
properties are satisfied, when in reality the system can get stuck unless
there are sufficient spurious wakeups.

We choose not to model the guards of alt branches, for we believe that doing
so would add a lot of complexity to the model, without adding much assurance.
The part of the implementation concerning guards is rather straightforward: a
branch whose guard is false is simply ignored.  We think it is best to
concentrate on the more difficult parts of the implementation.

In the model of a channel, the registration and deregistration functions are
wrapped in suitable |begin| and |end| events.  In the model of an alt, the
alt-thread performs these |begin| and |end| events, and then reacts to the
result in the |end| event.  Later, we combine the models of the alt and
channels together, synchronising on these events, so as to achieve the desired
effect.

Similarly, in the model of an alt, the callback functions are wrapped in
suitable |begin| and |end| events.  In the model of a channel, the
channel-thread performs these events, and reacts to the result in the |end|
event.  

Each use of the alt of alt-thread~|t| is framed with events |beginAlt.t| and
|endAlt.t.result| where |result| is of one of the following forms.
%
\begin{itemize}
\item |AltSend.i.x|, representing the sending of value~|x| on the port
  corresponding to the branch with index~|i|;

\item |AltReceive.i.x|, similarly representing receiving of~|x|;

\item |AltAbort|, representing an \SCALA{AltAbort} exception.
\end{itemize}
