\section{Conclusions}
\label{sec:conc}

In this paper we have analysed a library of communication primitives, using
CSP and its model checker FDR\@.  We have shown how the properties of
synchronisation linearisability and synchronisation progressibility can be  
captured as CSP refinement checks.  The analysis revealed an error in the
code.  We have shown how compositional verification can be used to make an
analysis more efficient, and so to allow larger systems to be analysed. 



We should be clear about what we have shown.  We have analysed particular
models, with particular numbers channel-threads, particular choices for the
branches of an alt, and a particular choice of the type |Data| of data
values.  In the absence of further arguments, these results to not imply that
corresponding results hold for larger values of those parameters would also
hold --- but they do help to give us confidence that they would, since
experience shows that must bugs are exhibited by rather small instances.  

The \emph{parameterised model checking problem} seeks to verify a family of
systems, for all values of certain parameters, such as those identified in the
previous paragraph.  The problem is undecidable in general~\cite{apt-kozen,
  tomasz-gavin-CA}.  Nevertheless, a number of approaches have been proposed,
that work in a number of situations.

The easier parameter to deal with is the type |Data| of data values.  The
models in this paper are data independent in this type: values are input,
nondeterministically choosen, stored, and output, but no operations are
applied to values that constrain the type.  Data independence has been much
studied previously, e.g.~\cite{Wolper-1986, ranko-thesis}: typically these
results show that if a correctness result holds for a particular size of a
type, they also hold for all larger types.  However, these do not give
useful results for the correctness properties in this paper: the results
of~\cite[Section~15.2]{awr:TPC}, when applied to the systems of
Section~\ref{sec:syncchan-timed} with five threads, would require us to use
\emph{eleven} data values, which is almost certainly infeasible.  In
Appendix~\ref{sec:di} we take a more direct approach, and prove that, in fact,
taking |Data| to include just \emph{two} data values is enough to deduce that
the results also hold for larger types.

There has been much work considering the parameterised model checking problem
where the parameter is the number of processes in the system,
e.g.~\cite{Lubachevsky:1984, Clarke:1987, wolper-lovinfosse89,
  sistla-german92, EN1995, Pnueli:2002, tomasz-gavin-CA, AHH16, gavin:VA22}.
A particular difficulty in applying these techniques to the work in this
paper is that the model of a monitor is parameterised by the set of identities
of threads that are currently waiting: to our knowledge, none of the
techniques can be applied in such a setting.



\bibliographystyle{alpha}
\bibliography{SCL-CSP}
