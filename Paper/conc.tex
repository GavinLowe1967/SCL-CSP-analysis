\section{Conclusions}
\label{sec:conc}

In this paper we have analysed a library of communication primitives, using
CSP and its model checker FDR\@.  We have shown how the properties of
synchronisation linearisability and synchronisation progressibility can be
captured as CSP refinement checks, and tested using FDR.  The specification is
built from a lineariser process for each thread: these processes synchronise
on events that represent the synchronisation points of operations.  

We believe that this style is widely applicable to other synchronisation
objects.  Consider a synchronisation object that allows binary
synchronisations between operations \SCALA{op1} and \SCALA{op2}.  We can use
events of the form |sync.t1.t2.x1.x2.y1.y2| to represent a synchronisation
between an execution by~|t1| of \SCALA{op1(x1)} returning~|y1|, and an
execution by~|t2| of \SCALA{op2(x2)} returning~|y2|.  (In many cases, like in
the body of this paper, the form of these events can be simplified.)  Then we
can define lineariser processes as follows.
%
\begin{cspm}
Lin(t) = 
  beginOp1.t?x1 -> sync.t?other!x1?x2?y1?y2 -> endOp1.t.y1 -> Lin(t) 
  [] beginOp2.t?x2 -> sync?other!t?x1!x2?y1?y2 -> endOp2.t.y2 -> Lin(t)
\end{cspm}
%
In each case, this thread contributes its identity and parameter to the |sync|
event, and receives back its return value.  The lineariser ensures that each
synchronisation event happens between the corresponding |begin| and |end|
events.  These linearisers can then be placed in parallel with a suitable
process that specifies what values should be returned, possibly as a function
of some state.  A generic definition for this process would be
%
\begin{cspm}
SyncSpec(state) = 
  sync?t1?t2?x1?x2!f1(state,x1,x2)!f2(state,x2,x2) -> SyncSpec(update(state,x1,x2))
\end{cspm}
%
where |state| captures the current state, |f1| and |f2| define the return
values as functions of the current state and the parameters, and |update|
defines how the state is updated.  These functions can be chosen so as to
capture the informal specification of the synchronisation object.

This construction can be generalised to allow an operation to synchronise in
multiple different ways, as in Section~\ref{sec:combined}.  It can also be
extended to allow synchronisations between more than two threads.

The analysis revealed an error in a previous version of the code.  But it also
helped to reach a correct implementation:I went through multiple iterations of
finding bugs using FDR, and coming up with possible fixes.  I don't think I
would have reached correct code without the benefits of model checking. 

We have shown how compositional
verification can be used to make an analysis more efficient, and so allow
larger systems to be analysed.
%
The compositional verification proved more difficult that we had expected.
One reason for this is that the concrete components have complex behaviours
which the idealised versions have to reflect.  Further, there is a decoupling
between the external interface ---in terms of |begin| and |end| events--- and
where those operation executions have an effect: the use of lineariser
components within the idealised models proved useful in capturing this
decoupling.  The other main source of difficulty was the necessity of
capturing environmental assumptions upon the components: we used error events
to signal a violation of an environmental assumption, which led to subsequent
arbitrary behaviour; we believe this technique works well.

We should be clear about what we have shown.  We have analysed particular
models, with particular numbers of channel-threads, particular choices for the
branches of an alt, and a particular choice of the type |Data| of data values.
In the absence of further arguments, these results to not imply that
corresponding results hold for larger values of those parameters would also
hold --- but they do help to give us confidence that they would, since
experience shows that most bugs are exhibited by rather small instances.

The \emph{parameterised model checking problem} seeks to verify a family of
systems, for all values of certain parameters, such as those identified in the
previous paragraph.  The problem is undecidable in general~\cite{apt-kozen,
  tomasz-gavin-CA}.  Nevertheless, several approaches have been proposed that
work in a number of situations.

The easier parameter to deal with is the type |Data| of data values.  The
models in this paper are data independent in this type: values are input,
nondeterministically chosen, stored, and output, but no operations are applied
to values that constrain the type.  Data independence has been much studied
previously, e.g.~\cite{Wolper-1986, ranko-thesis}: typically these results
show that if a correctness property holds for a particular size of a type, it
also holds for all larger types.  However, these results are not useful for
the correctness properties in this paper: the results
of~\cite[Section~15.2]{awr:TPC}, when applied to the systems of
Section~\ref{sec:syncchan-timed} with five threads, would require us to use
\emph{eleven} data values, which is almost certainly infeasible.  In
Appendix~\ref{sec:di} we take a more direct approach, and prove that, in fact,
taking |Data| to include just \emph{two} data values is enough to deduce that
the results also hold for larger types.

There has been much work considering the parameterised model checking problem
where the parameter is the number of processes in the system,
e.g.~\cite{Lubachevsky:1984, Clarke:1987, wolper-lovinfosse89,
  sistla-german92, EN1995, Pnueli:2002, tomasz-gavin-CA, AHH16, gavin:VA22}.
A particular difficulty in applying these techniques to the work in this paper
is that the model of a monitor is parameterised by the set of identities of
threads that are currently waiting, which is potentially unbounded: to our
knowledge, none of the existing techniques can be applied in such a setting.
We leave consideration of this question as future work. 

\subsection*{Acknowledgements}

This work has benefited from discussions with Jonathan Lawrence, Zeyang Zhao,
and Ilker Cicek.


\bibliographystyle{alpha}
\bibliography{SCL-CSP}
