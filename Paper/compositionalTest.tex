\section{Compositional verification}

\inlineCSP

We now consider an alternative approach to analysing the combination of an alt
and some channels.
%
\begin{enumerate}
\item We show that a single channel (in isolation) refines a more abstract
  model, which we call |IdealisedChannel|;

\item We likewise show that an alt (in isolation) refines a more abstract
  model, which we call |IdealisedAlt|;

\item We show that the combination of an |IdealisedAlt| and several
  |IdealisedChannel|s satisfies a property similar to that in the previous
  section. 
\end{enumerate}
%
This allows us to deduce that the combination of an alt and several channels
satisfies the same property. 

The analysis is complicated by the following issue.  A channel works correctly
under the assumption that any alt that interacts with it follows the alt
protocol.  However, if the alt does not follow the protocol, then the channel
can act incorrectly.  For example, if an alt tries to register \emph{twice}
with a channel, or tries to deregister without having previously registered,
then the implementation throws an exception, and the CSP model diverges.
Likewise, the alt works correctly under the assumption that channels follow
the alt protocol, but may act incorrectly otherwise.

When we analyse a channel in isolation, we cannot assume that its environment
(an alt) follows the protocol; and when we analyse an alt in isolation, we
cannot assume that its environment (channels) follows the environment: to make
these assumptions would be circular reasoning. 

Our approach is as follows.  Our idealised model of a channel will detect if
its environment breaks the alt protocol, and if so allow arbitrary behaviour.
Thus testing whether the model of a channel refines this specification is
equivalent to testing whether it satisfies the specification when the
environment does follow the protocol.  More precisely, we will produce a
process |ChannelSpec| that describes allowed behaviour when the environment
follows the protocol, but produces an \emph{error event} from a set
|Errors|$_C$ if the environment breaks the protocol.  We then test whether the
channel refines the process
%
\begin{cspm}
(ChannelSpec [|Errors£$_C$£|> Any) \ Errors£$_C$£
\end{cspm}
%
where |Any| allows arbitrary behaviour (the definition depends upon the
semantic model we are using).  

Likewise, we build an idealised model of an alt that allows arbitrary
behaviour if its environment breaks the protocol.  The model will be of the
form 
%
\begin{cspm}
(AltSpec [|Errors£$_A$£|> Any) \ Errors£$_A$£
\end{cspm}
%
where |AltSpec| describes allowed behaviour when the environment follows the
protocol.  

We can then consider the combination of |AltSpec| and several instances of
|ChannelSpec|.  Part of the analysis will show that no events from
$\CSPMM{Errors}_C \union \CSPMM{Errors}_A$ occur: each component follows the
alt protocol, provided the other does.  But we can also show that this
combination satisfies a property similar to that in the previous section,
allowing us to deduce that the corresponding combination of an alt and
channels satisfies that property. 

%%%%%%%%%%%%%%%%%%%%%%%%%%%%%%%%%%%%%%%%%%%%%%%%%%%%%%%
