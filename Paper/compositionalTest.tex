\section{Compositional verification}
\label{sec:compositional}

\inlineCSP

We now consider an alternative approach to analysing the combination of an alt
and some channels.  
%
\begin{enumerate}
\item\label{item:channel-idealise}
  We show that a single channel is consistent with a more
  abstract model, which we call |IdealisedChannel|;

\item\label{item:alt-idealise}
  We likewise show that an alt  is consistent with a more
  abstract model, which we call |IdealisedAlt|;

\item We show that the combination of an |IdealisedAlt| and several
  |Idealised|\-|Channel|s satisfies the correctness property from the previous
  section.
\end{enumerate}
%
This allows us to deduce that the combination of an alt and several channels
satisfies the same property.  The idea is that the ``idealised'' models,
|IdealisedChannel| and |IdealisedAlt|, capture the desired behaviour of the
components, abstracting away from their implementations. 

The fundamental reason why such an approach works is that CSP refinement is
compositional: if $P \refinedby Q$ then $C(P) \refinedby C(Q)$ for any
context~$C$ defined using CSP operators~\cite{awr:UCS}.  This means that if we
refine one component of a system, we also refine the whole system.
Alternatively, if we show that a component refines some specification, we can
replace that component by its specification in the analysis of the whole
system. 

This approach scales better than that in the previous section, so allows us to
deduce correctness for a larger number of threads.

The analysis in step~\ref{item:channel-idealise} considers the channel in
isolation, in the sense that we do not model the alt explicitly.  Instead, we
treat the alt as part of the environment of the channel.  This means we model
interactions between the channel and the alt, at the interface, without
modelling how the alt deals with those interactions.  Likewise, the analysis
in step~\ref{item:alt-idealise} considers the alt in isolation, without
modelling the channel explicitly.

The analysis is complicated by the following issue.  A channel works correctly
under the assumption that any alt that interacts with it follows the alt
protocol.  However, if the alt does not follow the protocol, then the channel
can act incorrectly.  For example, if an alt tries to register \emph{twice}
with a channel, or tries to deregister without having previously registered,
then the implementation throws an exception, and the CSP model diverges.
Likewise, the alt works correctly under the assumption that channels follow
the alt protocol, but it may act incorrectly otherwise.

When we analyse a channel in isolation, we cannot assume that its environment
(an alt) follows the protocol; and when we analyse an alt in isolation, we
cannot assume that its environment (channels) follows the environment: to make
these assumptions would be circular reasoning. 

The issue of components of a system making assumptions about other components
has been widely considered before, for example using the techniques of
rely-guarantee~\cite{jones-83} and
assumption-commitment~\cite{misra-chandy-81,PJ-91}.  Our approach shows how to
use CSP model checking in such a situation. 

We tackle this issue as follows.  Our idealised model of a channel detects if
its environment breaks the alt protocol, and if so allows arbitrary behaviour.
Thus testing whether the model of a channel refines this specification is
equivalent to testing whether it satisfies the specification when the
environment does follow the protocol.  More precisely, we produce a
process |IdealisedChannel| that describes allowed behaviour when the
environment does follow the protocol, but produces an \emph{error event} from
a set |Errors|$\CSPcolour_C$ if the environment breaks the protocol.

We then test (Section~\ref{sec:idealisedChan}) whether the channel model (from
Section~\ref{sec:syncchan}) refines the process
%
\begin{cspm}
ChannelSpec = (IdealisedChannel [|Errors£$\CSPcolour_C$£|> Any) \ Errors£$\CSPcolour_C$£
\end{cspm}
%
where |Any| allows arbitrary behaviour.  Thus the above process acts
like~|Any| after an error event.  The definition of |Any| defends upon the
semantic model we are using: in the stable-failures model, the appropriate
process is |CHAOS(Interface)|, where |Interface| is the interface of the
channel; in the failures-divergences model, the appropriate process is |DIV|.
Showing that the channel refines |ChannelSpec| will show that it acts like
|IdealisedChannel|, provided its environment follows the alt protocol.

\ifJournal{The construction of the \CSPM{IdealisedChannel} process is rather
  involved.  We therefore just sketch the construction here, and refer the
  reader to~\cite{TR} for full details.}

Likewise, we build a process |IdealisedAlt|, which describes allowed behaviour
when the environment does follow the protocol, but performs an event
from~|Errors|$\CSPcolour_A$ if the environment breaks the protocol.  We then
build an idealised model of an alt that allows arbitrary behaviour if its
environment breaks the protocol (Section~\ref{sec:idealisedAlt}), as follows:
%
\begin{cspm}
AltSpec = (IdealisedAlt [|Errors£$\CSPcolour_A$£|> Any) \ Errors£$\CSPcolour_A$£
\end{cspm}
%
%% where |IdealisedAlt| describes allowed behaviour when the environment does
%% follow the protocol, but performs an event from~|Errors|$\CSPcolour_A$ if the
%% environment breaks the protocol. 
Showing that the model of an alt (from Section~\ref{sec:alt}) refines
|AltSpec| will show that it acts like |IdealisedAlt|, provided its environment
follows the alt protocol.  
%
\ifJournal{As with the idealised channel, we just sketch the construction of
  \CSPM{IdealisedAlt} here, and refer the reader to~\cite{TR} for full
  details.}

We then consider the combination of |IdealisedAlt| and several instances of
|IdealisedChannel|, and test (Section~\ref{sec:combiningIdealised}) whether it
refines the specification |Spec| from Section~\ref{sec:combined}.

The following proposition generalises this style of compositional
verification. 
%
\begin{prop}
\label{prop:rely-guarantee}
Let $M$ be a CSP model, either {\it T} (traces), {\it F} (stable failures), or
{\it FD} (failures-divergences).  Let $P_i$, for $i \in I$, be a collection of
processes, with alphabets~$\alpha_i$.  For each $i$, let $Err_i$ be a set of
fresh ``error'' events (disjoint from each other and from each~$\alpha_j$),
let $Idealised_i$ be a process with alphabet $\alpha_i \union Err_i$, and let
\begin{eqnarray*}
Spec_i & = & (Idealised_i \throw{Err_i} {Any}_M) \hide Err_i,
\end{eqnarray*} 
where $Any_T = Any_F = \CSPMM{CHAOS}(\alpha_i)$ and $Any_{\it FD} =
\CSPMM{DIV}$.
%
Let $Spec$ be a process with alphabet $\Union_{i \in I} \alpha_i$ (so without
any error events), and let $A$ be a set of events disjoint from $\Union_{i \in
} Err_i$.  Suppose
\begin{enumerate}
\item\label{rg-1} $Spec_i \refinedby_M P_i$, for each~$i \in I$;

\item\label{rg-2} $Spec \refinedby_M 
 \left( \Parallel i:I \spot [\alpha_i \union Err_i] \, Idealised_i \right)
 \hide A$. 
\end{enumerate}
Then
\begin{eqnarray*}
Spec & \refinedby_M & 
  \left( \Parallel i:I \spot [\alpha_i] \, P_i \right) \hide A.
\end{eqnarray*}
\end{prop}
%
The idea is that, for each~$i$,\, $Idealised_i$ is an idealised model
of~$P_i$, which describes $P_i$'s behaviour under the assumption that its
environment satisfies certain assumptions, but performs an error event
otherwise; so $Spec_i$ allows arbitrary behaviour when the environment does
not follow the assumptions.

\begin{proof}
From condition~\ref{rg-2}, and the assumption that~$Spec$ performs no error
event from $\Union_{i \in I} Err_i$, we can deduce that $\Parallel i:I \spot
[\alpha_i \union Err_i] \, Idealised_i$ performs no error event.  But this
implies that $\Parallel i:I \spot [\alpha_i] \, Spec_i$ likewise performs no
(hidden) error event, since the two processes behave identically up to the
first error event.  So, in fact,
\begin{eqnarray*}
\Parallel i:I \spot [\alpha_i \union Err_i] \, Idealised_i & = & 
  \Parallel i:I \spot [\alpha_i] \, Spec_i,
\end{eqnarray*}
since the ``throws'' operator in each $Spec_i$ is never triggered.  Thus
\[
Spec  \;\refinedby_M\; 
  \left( \Parallel i:I \spot [\alpha_i \union Err_i] \, Idealised_i \right)
    \hide A
  \; = \; 
  \left(  \Parallel i:I \spot [\alpha_i] \, Spec_i \right) \hide A
  \;\refinedby\; 
  \left( \Parallel i:I \spot [\alpha_i] \, P_i \right) \hide A,
\]
using conditions~\ref{rg-1} and~\ref{rg-2}, and the fact that refinement is
compositional with respect to CSP operators (specifically parallel composition
and hiding).
\end{proof}


We can apply the above proposition, instantiating the $P_i$ with a model of a
channel or alt implementation, instantiating the $Idealised_i$ with
|IdealisedChannel| or |IdealisedAlt|, instantiating the $Spec_i$ with
|ChannelSpec| or |AltSpec|, and instantiating $Spec$ with the specification
from Section~\ref{sec:combined}.  Condition~\ref{rg-1} of the proposition is
discharged via the refinement checks, concerning |ChannelSpec| and |AltSpec|,
described above.  Condition~\ref{rg-2} can be discharged by combining
|IdealisedChannel| and |IdealisedAlt| processes, in the same way that we
combined channel and alt processes in Section~\ref{sec:combined}.

%%%%%%%%%%

%% Part of the analysis
%% (Section~\ref{sec:combiningIdealised}) will show that no events from
%% $\CSPMM{Errors}\CSPcolour_C \union \CSPMM{Errors}\CSPcolour_A$ occur: each component follows the
%% alt protocol, provided the other has not previously broken the protocol.  But
%% we can also show that this combination satisfies the correctness property from
%% the previous section, allowing us to deduce that the corresponding combination
%% of an alt and channels satisfies that property.

