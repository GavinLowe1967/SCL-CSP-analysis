\subsection{Combining the idealised models}

We now perform the final step in our compositional verification by combining
an |IdealisedAlt| and corresponding |IdealisedChannel|s.
%
Fix a definition of |branches|, defining the branches of an
alt.  We can then build a system comprising an |IdealisedAlt| (based on
|branches|) and a corresponding set of |IdealisedChannel|s, synchronising
appropriately.

\begin{window}[1,r,{
%
% \vspace*{1mm}
\vspace{0.5ex}
\begin{tabular}{\|cccc\|}
Model & Threads & States  & Time\\
F & 4 & 12.2M & 19s \\
D & 4 & 5.53M & 24s \\
F & 5 & 909M & 1820s \\
D & 5 & 227M & 1710s \\
\end{tabular}
},]
%
We can then use FDR to verify that this system is divergence-free (when the
|spuriousWakeup| events are kept visible), and that it refines (in the
stable-failures model) the specification from Section~\ref{sec:combined} (with
the |spuriousWakeup| events hidden).  The table to the right gives statistics about these checks, where
|branches| contains one inport branch and one output branch.
\end{window}

In particular, the above stable-failures test verifies that the system does
not perform any of the error events from |Errors|$_A \union\null$|Errors|$_C$:
each component follows the alt protocol assuming the other has not previously
broken the protocol.

Recall that the processes |ChannelSpec| and |AltSpec| behaved like
|IdealisedChannel| and |IdealisedAlt|, except that they allowed arbitrary
behaviours after an error event.  The fact that the combination of
|IdealisedChannel|s and |IdealisedAlt| does not perform any error events means
that the same combination of |ChannelSpec|s and |AltSpec| would not perform
and error events, and so would behave identically overall.  That means that
the combination would also pass the tests in the previous paragraph (although
the checks would take slightly longer).

Further, we earlier showed that the models of the implementations refine
|ChannelSpec| and |AltSpec|.  This then implies that the same combination of
the implementations would also pass the tests described above (although the
checks might be infeasible in practice): formally, this is because all CSP
operators are monotonic with respect to refinement.

%% \framebox{Type sizes, timing}

%% However, we can make the analysis more efficient.  The idealised model of a
%% synchronous channel contains events representing synchronisations between two
%% channel threads; these events are on a CSP channel |sync|, which are hidden.
%% Likewise, the specification from Section~\ref{sec:combined} contains events
%% representing these synchronisations; these events are also on a channel called
%% |sync|, and are hidden (the two |sync| channels are in different CSP modules,
%% so are distinct).  Instead, we could leave these events visible, and apply a
%% renaming to identify them.  We can then test for refinement between the
%% resulting processes: this refinement implies the refinement tested for
%% earlier.  This new refinement test explores fewer states, because the two
%% processes are more tightly coupled: each performs the same |sync| events.

%% We can extend this idea.  Each of the processes contain events representing
%% the linearisation point of a channel being closed (on hidden |close|
%% channels), and a timed operation timing out (on hidden |timeout| channels); we
%% can again keep these events visible and identify them.  

%% We can perform a similar transformation for the synchronisations between an
%% alt-thread and a channel-thread, on hidden |altSync| channels.  However, the
%% transformed specification doesn't quite capture the desired property.
%% Consider a state where an alt-thread could synchronise with either of two
%% channel-threads.  In the implementation model, the choice of which
%% synchronisation happens is made internally, i.e.~nondeterministically.
%% Likewise, in the earlier specification, the choice of which synchronisation
%% happens corresponds to a choice between two internal |altSync| events, so is
%% nondeterministic.  But when we keep these events visible, the choice is
%% external (corresponding to the use of an external choice in the |AltLin|
%% processes), which means that the specification requires that \emph{both}
%% |altSync| events are simultaneously available.  This means that the refinement
%% test will not hold.   

%% \begin{tabular}{\|cccc\|}
%% Model & Threads & States  & Time\\
%% F & 4 &  6.8M & 15.4s \\
%% F & 5 & 247M & 751s
%% \end{tabular}

%% \framebox{!!!}  I don't think this is sound.  In a state where there are two
%% possible synchronisations between channel threads, the idealised channel model
%% with |sync| events visible will give an \emph{external} choice between the
%% |sync| events.  However, the implementation model will make a nondeterministic
%% choice between which synchronisation happens.  The idealised channel model
%% with visible |sync| events does not correctly reflect the behaviour of the
%% implementation.
