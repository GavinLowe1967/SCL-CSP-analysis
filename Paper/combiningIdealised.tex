\subsection{Combining the idealised models}

We now perform the final step in our compositional verification by combining
an |IdealisedAlt| and corresponding |IdealisedChannel|s.

Fix, for the moment, a definition of |branches|, defining the branches of an
alt.  We can then build a system comprising an |IdealisedAlt| (based on
|branches|) and a corresponding set of |IdealisedChannel|s, synchronising
appropriately.

We can then use FDR to verify that this system is divergence-free (when the
|spuriousWakeup| events are kept visible), and that it refines (in the
stable-failures model) the specification from Section~\ref{sec:combined} (with
the |spuriousWakeup| events hidden).  In particular, the latter test verifies
that the system does not perform any of the error events
from~$\CSPMM{Errors}_A \union \CSPMM{Errors}_C$: each component follows the
alt protocol assuming the other does. 

Recall that we earlier considered processes |ChannelSpec| and |AltSpec|, which
allowed arbitrary behaviours after an error event.  The fact that the
combination of |IdealisedChannel|s and |IdealisedAlt| does not perform any
error events means that the same combination of |ChannelSpec|s and |AltSpec|
would behave identically; so that combination would also pass the tests in the
previous paragraph (although the checks would take slightly longer).

Further, we earlier showed that the models of the implementations refine
|ChannelSpec| and |AltSpec|.  This then implies that the same combination of
the implementations would also pass the tests described above (although the
checks might be infeasible in practice): formally, this is because all CSP
operators are monotonic with respect to refinement.

\framebox{Type sizes, timing}
