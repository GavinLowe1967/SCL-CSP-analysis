\section{Modelling and analysing a  synchronous channel}
\label{sec:syncchan}

\inlineScala

In this section, we consider the operation of a single synchronous channel,
without considering interaction with alts.

We start by considering just the send and receive operations.  We outline the
implementation for these operations, how we model them in CSP, and then how we
analyse their correctness.  In later subsections, we extend to include the
timed operations and the closing of channels; for the moment we elide details
related to those operations or to the use of alts.  Recall that channels are
shared: multiple senders and receivers can compete to use the channel. 

The implementation is based on a monitor (more precisely, using an
implementation of monitors within the SCL library).  All operations are
carried out while holding the monitor's lock.  In addition, the implementation
uses a number of \emph{conditions}: one thread can wait on a condition until
another thread signals on the same condition.

The implementation uses a variable \SCALA{value} to store the value that a
thread is currently trying to send (if any).  Further, it uses a variable
\SCALA{status} to store the status of the current exchange, one of the
following: 
\begin{description}
\item[{\scalastyle Empty}:] No sender has deposited a value;
\item[{\scalastyle Filled}:] A sender has deposited a value, but no receiver
  has yet read it;
\item[{\scalastyle Read}:] A receiver has read the current value, but the
  corresponding sender has not yet returned.
\end{description}
%
The implementation also uses a variable |receiversWaiting|, which records how
many receivers are currently waiting to receive a value. 

%%%%%

\begin{figure}
\begin{center}
\def\rx{3} % x coord of receiver
\def\height{4}  % height of fig
\def\delta{0.8} % vertical gap between messages
\begin{tikzpicture}[yscale = 1.0, >= angle 60]
\draw (0,0) node[draw](sender){Sender};
\draw (sender) -- ++ (0,-\height);
\draw (\rx,0) node[draw](receiver){Receiver};
\draw (receiver) -- ++ (0,-\height);
\draw (\rx+\rx,0) node[draw](next-sender){Next sender};
\draw (next-sender) -- ++ (0,-\height);
\draw[<-] (0,-1) -- node[above]{\scalashape\small slotEmptied} ++ (-\rx, 0);
\draw[->] (0,-1-\delta) -- node[above]{\scalashape\small slotFull} ++ (\rx, 0);
\draw[<-] (0,-1-2*\delta) -- node[above]{\scalashape\small continue} ++ (\rx, 0);
\draw[->] (0,-1-3*\delta) -- 
  node[above, near start]{\scalashape\small slotEmptied} ++ (\rx+\rx, 0);
\end{tikzpicture}
\end{center}
\caption{Sequence diagram illustrating signalling on conditions within a channel
  implementation. \label{fig:channel-sd}} 
\end{figure}

%%%%%

Figure~\ref{fig:channel-sd} illustrates the use of conditions within the
channel implementation. 
A sender waits (on a condition \SCALA{slotEmptied}) until the status is
\SCALA{Empty}.  It then stores its value in |value|, sets the status to
|Filled|, and signals (on a condition |slotFull|) to a receiver.  It next 
waits (on a condition |continue|) until the status is |Read|.  Finally, it
sets the status back to |Empty|, and signals (on |slotEmptied|) to the next
sender.

A receiver waits (on |slotFull|) until the status is |Filled| (updating
|receiversWaiting| before and after).  It then sets the status to |Read|,
signals to the sender (on |continue|), and returns the value stored in
|value|.

We now outline the CSP model.  The model uses small fixed types representing
the type of data communicated by the channel, and the type of thread
identities.  (We discuss the choices for these types in
Section~\ref{sec:conc}.)  

The monitor is modelled by a CSP module, encapsulating a process that
maintains the state of the monitor, specifically: (1)~which thread, if any,
holds the lock on the monitor; and (2)~which threads are waiting on which
conditions.  When no thread holds the lock, the monitor process allows a
thread, other than any of the waiting threads, to acquire the lock.  The
thread that currently holds the lock can:
\begin{itemize}
\item Release the lock;
\item Wait on a condition, at which point it also releases the lock;
\item Signal on a condition, at which point a thread that it waiting on that
  condition is recorded as no longer waiting (but needs to reacquire the lock
  before it can continue).
\end{itemize}
%
The monitor module provides an interface to client processes corresponding to
the different monitor operations.

A channel is also modelled by a CSP module, encapsulating the processes
described in the following paragraphs.   

Each variable in the implementation of the channel is modelled by a CSP
process, with CSP channels to allow the variable's value to be read or
written (see Appendix~\ref{app:modelling}).

Each of the send and receive operations can then be straightforwardly
translated into a CSP process: this process performs the operations on the
monitor and variables, following the Scala code.
(Appendix~\ref{app:modelling} describes various techniques that we found
useful in this modelling.)

The Scala code uses several assertions.  We model these by having the CSP
process diverge if the property does not hold.  Later, we use FDR to verify
that such divergences do not occur.  

Each operation is framed by \CSPM{begin} and \CSPM{end} events:
%
\begin{itemize}
\item The event \CSPM{beginSend.t.x} represents a thread with
  identity~\CSPM{t} calling |send(x)| on the channel, and the event
  \CSPM{endSend.t.SendSuccess} represents it successfully returning (later we
  add events to model unsuccessful calls that find the channel has
  been closed).

\item Likewise the events \CSPM{beginReceive.t} and
  \CSPM{endReceive.t.ReceiveSuccess.x} represent thread~\CSPM{t} calling and
  returning from an invocation of the |receive| operation that successfully
  receives the value~\CSPM{x}.
\end{itemize}
%
Inside the channel module is a process corresponding to each thread, which
accepts the \CSPM{begin} events, simulates the operation, and then performs
the appropriate \CSPM{end} event.  The module encapsulates this so only the
\CSPM{begin} and \CSPM{end} events are visible.  A process outside the module
can simulate calling the operation by synchronising on those events.

%%%%%%%%%%

\subsection{Analysing the basic channel}
\label{sec:syncchan-analysis-1}

We now analyse the basic model of the channel.  We start by describing
abstractly the property that we expect to hold, and then describe how to
capture it using CSP.

Each successful invocation of the send operation should synchronise with a
corresponding invocation of receive: the two invocations should overlap in
time, and the receive should return the value of the send's parameter.

In~\cite{LL:synchronisation}, we introduced a correctness condition,
\emph{synchronisation linearisation}, corresponding to synchronisation objects
like this.  The idea is that each synchronisation should appear to take place
between the begin and end of the two corresponding invocations; different
synchronisations should occur in a one-at-a-time order.  We call the points at
which the synchronisations appear to take place \emph{synchronisation points}.
(The definition is based on \emph{linearisation}~\cite{herlihy-wing}, the
standard correctness property for concurrent datatypes, where invocations of
operations should appear to take place in a one-at-a-time order, with each
between the begin and end of that invocation.)

The diagram below illustrates the idea; it illustrates a particular history
of a channel (or a trace of the CSP model).
%
%\begin{figure}
\begin{center}
\unScalaMid
\begin{tikzpicture}[xscale = 0.9]
\draw[|-|] (0,0) -- node[above] {$\sm{send}(A)$} (4,0);
\draw[|-|] (4.5,0) -- node[above] {$\sm{send}(B)$} (8.5,0);
\draw[|-|] (0.5,-1) -- node[above] {$\sm{send}(A)$} (8,-1);
\draw[|-|] (1,-2) -- node[above] {$\sm{receive}\mathord{:}A$} (3,-2);
\draw[|-|] (3.5,-2) -- node[above] {$\sm{receive}\mathord{:}B$} (6.5,-2);
\draw[|-|] (7,-2) -- node[above] {$\sm{receive}\mathord{:}A$} (9,-2);
\draw (1.9,0) \X; \draw (1.9,-2) \X; 
\draw[dashed] (1.9,0) -- (1.9,-2); % sync 1 and 3
\draw (5.6,0) \X; \draw (5.6,-2) \X; 
\draw[dashed] (5.6,0) -- (5.6,-2); % sync 4 and 5
\draw (7.5,-1) \X; \draw (7.5,-2) \X; 
\draw[dashed] (7.5,-1) -- (7.5,-2); % sync 2 and 6 
\end{tikzpicture}
\scalaMid
\end{center}
%% \caption{Timeline representing the synchronisation example.}
%% \label{fig:sync-timeline}
%% \end{figure}
%
Time goes from left to right in the diagram.  Each horizontal line represents
an invocation, with the end points representing the call and return (or
\CSPM{begin} and \CSPM{end} events).  The ``$\cross$''s in the diagram
illustrate the synchronisation points of the corresponding invocations, linked
by a dashed vertical line.

Thus a history (or trace) is synchronisation linearisable if it is possible to
identify synchronisation points with the desired properties.  This corresponds
to identifying which invocations synchronise with one another.  A
synchronisation object is synchronisation linearisable if all of its histories
are synchronisation linearisable~\cite{LL:synchronisation}.

%% The current model of the channel is (in the terminology
%% of~\cite{LL:synchronisation}) \emph{stateless}: no state is carried forward
%% from one synchronisation to another.  However, when we consider closing of the
%% channel, it will become \emph{stateful}, with two states, open or closed.
%% Other synchronisation objects have more interesting states.  \framebox{later?}

\inlineCSP

We now describe how we use model checking to test synchronisation
linearisability of the channel.  We create an instance |C| of the channel
module.  Below, expressions of the form |C::v| represent a value~|v| from this
instance.  We create a process |System| that combines the channel and a number
of threads (with identities from a set |ThreadID|): each thread repeatedly
calls the send and receive operations (represented by events on CSP channels
|C::beginSend| and |C::beginReceive|), and waits for them to return (on CSP
channels |C::endSend| and |C::endReceive|).

We build a CSP specification process that allows precisely the traces that are
synchronisation linearisable.  We use events of the form
\CSPM{sync.t}$_1$\CSPM{.t}$_2$\CSPM{.x} to represent a synchronisation point
between sender~$\CSPMM{t}_1$ and receiver~$\CSPMM{t}_2$, passing data
value~|x|.  We build a \emph{lineariser} process for each thread as follows,
which ensures that the |sync| events occur between the corresponding |begin|
and |end| events.
%
\begin{cspm}
Lin(t) = 
  C::beginSend.t?x -> sync.t?other!x -> C::endSend.t.SendSuccess -> Lin(t)
  [] C::beginReceive.t -> sync?other!t?x -> C::endReceive.t.ReceiveSuccess.x -> Lin(t)
\end{cspm}
%
When sending, the thread can synchronise with any other thread~|other|,
passing its value~|x|.  When receiving, the thread can synchronise with any
other thread, accepting that thread's value~|x|; this |x| is subsequently
returned by the invocation.

We combine the |Lin| processes in parallel, with their natural
alphabets, so the linearisers for threads $\CSPMM{t}_1$ and $\CSPMM{t}_2$
synchronise on events of \CSPM{sync.t}$_1$\CSPM{.t}$_2$ and
\CSPM{sync.t}$_2$\CSPM{.t}$_1$.  
%
\begin{cspm}
alphaLin(t) =
  {|C::beginSend.t, C::endSend.t, C::beginReceive.t, C::endReceive.t|} £$\union$£
  {|sync.t.other, sync.other.t | other <- ThreadID-{t}|}
Spec£$_0$£ = **|| t <- ThreadID @ [alphaLin(t)] Lin(t)
\end{cspm} % £$\union$£
%
Thus each trace of |Spec|$_0$ represents a history that is synchronisation
linearisable, together with the corresponding synchronisation points.  By
hiding the synchronisation points, we obtain a process whose traces represent
precisely those histories that are synchronisation linearisable.  We can then
use FDR to check that the system is synchronisation linearisable. 
%
\begin{cspm}
Spec = Spec£$_0$£ \ {|sync|}
assert Spec [T= System
\end{cspm}

We can also test the same refinement in the stable failures model (which
implies the refinement in the traces model).
%
\begin{cspm}
assert Spec [F= System
\end{cspm}
%
This captures a useful progress property, which says that if a synchronisation
is possible (i.e.~there is at least one |send| and one |receive| operation
that have been called but not yet returned), then some such synchronisation
must happen (since an internal |sync| event is available), and so the relevant
threads reach a state where they can return.  (If several different
synchronisations are possible, then the refinement test allows any to happen.)
Informally, threads don't get stuck unnecessarily.  We call this property
\emph{synchronisation progressiblity}~\cite{LL:synchronisation}.

Finally, as discussed above, we can test that |System| does not diverge, to
verify that no thread in the implementation throws an exception.  Further,
this verifies that threads cannot perform an infinite amount of internal
activity without any thread returning.  This test can be combined with the
previous by testing for refinement in the failures-divergences model.
%
\begin{cspm}
assert Spec [FD= System
\end{cspm}

