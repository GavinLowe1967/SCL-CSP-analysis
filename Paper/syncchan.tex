\section{Modelling and analysing a  synchronous channel}
\label{sec:syncchan}

\inlineScala

In this section, we consider the operation of a single synchronous channel,
without considering interaction with alts.
%
We start by considering just the send and receive operations: we outline the
implementation for these operations, how we model them in CSP, and then how we
analyse their correctness.  We then extend the analysis to include the timed
operations and the closing of channels.  Recall that channels are shared:
multiple senders and receivers can compete to use the channel.


%%%%%%%%%%%%%%%%%%%%%%%%%%%%%%%%%%%%%%%%%%%%%%%%%%%%%%%

\subsection{A simplified channel}

In this section, we describe the implementation of the send and receive
operations.  For the moment, we elide the parts of the implementation related
to timed operations, the closing of channels, and alts: we describe a
stripped-down implementation with those aspects removed.

%% We start by considering just the send and receive operations.  We outline the
%% implementation for these operations, how we model them in CSP, and then how we
%% analyse their correctness.  In later subsections, we extend the analysis to
%% include the timed operations and the closing of channels; for the moment we
%% elide details related to those operations or to the use of alts.  Recall that
%% channels are shared: multiple senders and receivers can compete to use the
%% channel.

The SCL implementation is based on a monitor (more precisely, using an
implementation of monitors within the SCL library), called |lock|.  All
operations are carried out while holding the monitor's lock.  In addition, the
implementation uses three \emph{conditions} within the monitor: one thread can
wait on a condition until another thread signals on the same condition.

The implementation uses a variable \SCALA{value} to store the value that a
thread is currently trying to send (if any).  Further, it uses a variable
\SCALA{status} to store the status of the current exchange, which is one of
the following values:
%
\begin{description}
\item[{\scalastyle Empty}:] No sender has deposited a value;
\item[{\scalastyle Filled}:] A sender has deposited a value, but no receiver
  has yet read it;
\item[{\scalastyle Read}:] A receiver has read the current value, but the
  corresponding sender has not yet returned.
\end{description}
%
%% The implementation also uses a variable |receiversWaiting|, which records how
%% many receivers are currently waiting to receive a value. 

%%%%%

\begin{figure}
\begin{minipage}[t]{80mm}
\begin{scala}
def send(x: A) = lock.mutex{
  while(status != Empty) slotEmptied.await()
  value = x; status = Filled; slotFull.signal()
  continue.await(); assert(status == Read)
  status = Empty; slotEmptied.signal()
}
\end{scala}
\end{minipage}
%
\begin{minipage}[t]{75mm}
\begin{scala}
def receive(): A = lock.mutex{
  while(status != Filled) slotFull.await()
  status = Read; continue.signal()
  value
}
\end{scala}
\end{minipage}

%%%%%

\begin{center}
\def\rx{3} % x coord of receiver
\def\height{4}  % height of fig
\def\delta{0.8} % vertical gap between messages
\begin{tikzpicture}[yscale = 1.0, >= angle 60]
\draw (0,0) node[draw](sender){Sender};
\draw (sender) -- ++ (0,-\height);
\draw (\rx,0) node[draw](receiver){Receiver};
\draw (receiver) -- ++ (0,-\height);
\draw (\rx+\rx,0) node[draw](next-sender){Next sender};
\draw (next-sender) -- ++ (0,-\height);
\draw[<-] (0,-1) -- node[above]{\scalashape\small slotEmptied} ++ (-\rx, 0);
\draw[->] (0,-1-\delta) -- node[above]{\scalashape\small slotFull} ++ (\rx, 0);
\draw[<-] (0,-1-2*\delta) -- node[above]{\scalashape\small continue} ++ (\rx, 0);
\draw[->] (0,-1-3*\delta) -- 
  node[above, near start]{\scalashape\small slotEmptied} ++ (\rx+\rx, 0);
\end{tikzpicture}
\end{center}
\caption{The code for sending and receiving, and a sequence diagram
  illustrating signalling on conditions within a channel
  implementation. 
\label{fig:channel-sd}}\label{fig:basic-channel-scala}
\end{figure}

%%%%%

Figure~\ref{fig:channel-sd} gives code for sending and receiving.  (Note: this
is a large simplification of the full code, which supports alts, closing of
the channel, and the timed operations; we give this stripped-down version to
illustrate the modelling and analysis technique.)  Each operation acts under
mutual exclusion on |lock| (denoted ``\SCALA{lock.mutex\{...\}}'').  The
figure also gives a sequence diagram to illustrate the use of conditions.

The sender waits on the condition \SCALA{slotEmptied} until the status is
\SCALA{Empty}.  It then stores its value in |value|, sets the status to
|Filled|, and signals on the condition |slotFull| to the receiver.  It next 
waits on the condition |continue| until the status is |Read|.  Finally, it
sets the status back to |Empty|, and signals on |slotEmptied| to the next
sender.

The receiver waits on |slotFull| until the status is |Filled|.  It then sets
the status to |Read|, signals to the sender on |continue|, and returns the
value stored in |value|.

%%%%%%%%%%%%%%%%%%%%%%%%%%%%%%%%%%%%%%%%%%%%%%%%%%%%%%%

\subsection{Basic CSP model}
\label{sec:basic-csp-model}

We now outline the CSP model; Figure~\ref{fig:basic-CSP-model} gives an
overview.  The model uses small fixed types |Data| representing the type of
data communicated by the channel, and |ThreadID| representing the type of
thread identities.  (We discuss the choices for these types in
Section~\ref{sec:conc}.)

%%%%%

\begin{figure}
\begin{center}
\def\gap{0.1}
\def\threadW{15mm}
\def\height{8mm}
\begin{tikzpicture}[> = triangle 90]
%%%%% Variables
\draw (0,0) node[draw, minimum height = \height] (status){\cspmstyle Status};
\draw (0,-1) node[draw, minimum height = \height] (value){\cspmstyle Value};
%%%%% Monitor
\draw (0,-3) node[draw] (monitor){%
  \cspmstyle
  \begin{tabular}{c}
  Monitor  \\ \cline{1-1}
  Lock \\
  Unlock \\
  Await \\
  Signal
  \end{tabular}};
%%%%% Threads
\draw (6,-1) node[draw, minimum width = \threadW, minimum height = \height]
  (thread){\cspmstyle Thread};
\draw (thread.north west) ++ (\gap, 0.0) |- ++ (\threadW, \gap) |- 
  ++ (-\gap, -\height);
%%%%% Channels
\draw (thread.north) ++ (0,\gap) -- (6,0) -- 
  node[above]{\cspmstyle getStatus, setStatus} (status); 
\draw (thread) -- node[above]{\cspmstyle getValue, setValue} (value); 
\draw[->] (thread) -- (6,-3) -- (monitor);
\draw (thread.east) ++ (\gap,0.0) -- node[above, near end]{\cspmstyle
  \begin{tabular}{c}
  beginSend, endSend, \\ beginReceive, endReceive
  \end{tabular}} ++ (3.5,0);
%
\draw (-1.4,1) -- (7.3,1) -- (7.3,-4.5) -- (-1.4,-4.5) -- (-1.4,1);
\end{tikzpicture}
\end{center}
\caption{Representation of the CSP model of a basic channel.  Channel
  communications are represented by lines labelled with the channel names;
  calls of functions provided by the {\cspmstyle Monitor} module are
  represented by a line with a triangular arrowhead.}
\label{fig:basic-CSP-model}
\end{figure}

%%%%%

We model each shared variable using a CSP process as follows.  Here |value| is
the current value of the variable, and |get| and |set| are channels on which a
thread~|t| can get or set the value.
%
\begin{cspm}
Var(value, get, set) = 
  get?t!value -> Var(value, get, set)
  [] set?t?value' -> Var(value', get, set)
\end{cspm}
%
We can then model the |value| and |status| variables is follows.
(We initialise |value| to a particular value~|A|.)
\begin{cspm}
channel getValue, setValue: ThreadID.Data
Value = Var(A, getValue, setValue) 
datatype StatusT = Empty | Filled | Read
channel getStatus, setStatus : ThreadID . StatusT
Status = Var(Empty, getStatus, setStatus)
\end{cspm}

%%%%%

The monitor |lock| is modelled by a CSP module |Monitor|.  This encapsulates a
process that maintains the state of the monitor, defined recursively in terms
of the following two families of processes:\footnote{%
%
The type \CSPM{(|ConditionID => \{ThreadID\}|)} represents maps from condition
identities to sets of thread identities.  The type \CSPM{ConditionID} contains
an identity for each of the conditions described earlier.}
%
\begin{cspm}
Unlocked :: ((|ConditionID => {ThreadID}|)) -> Proc 
Locked :: (ThreadID, (|ConditionID =>{ThreadID}|)) -> Proc 
\end{cspm}
%
(We omit here some details, which we will need to model timed operations in
Section~\ref{sec:syncchan-timed}.)  The process |Unlocked(waiting)| represents
that the monitor is currently unlocked; the map |waiting| records, for each
condition~$c$, the set of threads currently waiting on~$c$.  The process
|Locked(t, waiting)| represents that the monitor is currently locked by
thread~|t|; |waiting| again records which threads are waiting on conditions.
In the |Unlocked| states, a thread~|t| can acquire the lock, via event
|acquire.t| (which moves the monitor process into a |Locked| state).  In the
|Locked| states, the current thread~|t| can
\begin{itemize}
\item Release the lock, via event |release.t| (which moves the monitor process
  into an |Unlocked| state);
\item Wait on a condition~|c|, via event |await.t.c| (which moves the monitor
  process into an |Unlocked| state, updating the |waiting| map appropriately);
\item Signal on a condition~|c|, via event |signal.t.c.t'|, to a thread |t'|
  currently waiting on~|c|, if there is one (|waiting| is updated
  appropriately); the thread~|t'| needs to reacquire the lock before it can
  continue.
\end{itemize}
%
However, for convenience, the CSP module exports four functions which can be
called by a thread~|t|:
%
\begin{itemize}
\item |Lock(t)|: acquire the lock, blocking until it is available;
\item |Unlock(t)|: release the lock;
\item |Await(t,c)|: wait on condition~|c|; when a signal is received, wait to
  reacquire the lock;
\item |Signal(t,c)|: signal to a thread on condition~|c| (if there is a
  waiting thread). 
\end{itemize}

%% %
%% Threads can synchronise with this process on channels |acquire|, |release|,
%% |await| and |signal| (with obvious meanings).  However, for convenience, the
%% CSP module exports the following functions

%% \begin{itemize}
%% \item \CSPM{Lock(t)}: Release the lock;
%% \item \CSPM{Unlock(t)}: 

%% Wait on a condition, at which point it also releases the lock;
%% \item Signal on a condition, at which point a thread that is waiting on that
%%   condition is recorded as no longer waiting (but needs to reacquire the lock
%%   before it can continue).
%% \end{itemize}

%% , encapsulating a process that
%% maintains the state of the monitor, specifically: (1)~which thread, if any,
%% holds the lock on the monitor; and (2)~which threads are waiting on which
%% conditions.  When no thread holds the lock, the monitor process allows a
%% thread, other than any of the waiting threads, to acquire the lock.  The
%% thread that currently holds the lock can:
%% \begin{itemize}
%% \item Release the lock;
%% \item Wait on a condition, at which point it also releases the lock;
%% \item Signal on a condition, at which point a thread that is waiting on that
%%   condition is recorded as no longer waiting (but needs to reacquire the lock
%%   before it can continue).
%% \end{itemize}
%% %
%% The monitor module provides an interface to client processes corresponding to
%% the different monitor operations.

%% The model of a channel includes a monitor, and various other processes.
%% Each variable in the implementation of the channel is modelled by a CSP
%% process, with CSP channels to allow the variable's value to be read or
%% written (see Appendix~\ref{app:modelling}).

%%%%%

\begin{figure}
\begin{cspm}
Send(me, x) = Monitor::Lock(me); Send1(me, x)

Send1(me, x) = 
  getStatus.me?s ->
  if s == Empty then Send2(me, x) else (Monitor::Await(me, SlotEmptied); Send1(me, x))

Send2(me, x) = 
  setValue.me.x -> setStatus.me.Filled -> Monitor::Signal(me, SlotFull);	
  Monitor::Await(me, Continue); 
  getStatus.me?s -> (if s == Read then SKIP else DIV);
  setStatus.me.Empty -> Monitor::Signal(me, SlotEmptied);
  Monitor::Unlock(me); endSend.me.SendSuccess -> SKIP

Receive(me) = Monitor::Lock(me); Receive1(me)

Receive1(me) =
  getStatus.me?s -> 
  if s != Filled then Monitor::Await(me, SlotFull); Receive1(me) else Receive2(me)

Receive2(me) = 
  setStatus.me!Read -> Monitor::Signal(me, Continue); 
  getValue.me?v -> (Monitor::Unlock(me); endReceive.me.ReceiveSuccess.v -> SKIP) 

ChannelThread(me) = 
  beginSend.me?x -> Send(me, x); ChannelThread(me)
  [] beginReceive.me -> Receive(me); ChannelThread(me)
\end{cspm}
\caption{CSP processes modelling the {\scalashape send} and {\scalashape
    receive} operations, and a {\scalashape ChannelThread} process that
  performs the operations.}
\label{fig:CSP-send-receive}
\end{figure}

%%%%%

Each of the send and receive operations can then be straightforwardly
translated into a CSP process: see Figure~\ref{fig:CSP-send-receive}.  This
process performs the operations on the monitor and variables, following the
Scala code.  Each expression of the form |Monitor::Op(args)| represents a call
to |Op(args)| within the |Monitor| module, which models~|lock|.

%% (Appendix~\ref{app:modelling} describes various techniques that
%% we found useful in this modelling.)

%% Unlike shared variables, each thread-local variable is modelled as a parameter
%% of the relevant process: this reduces the number of events performed by
%% processes, and so reduces the state-space of the system to be checked;
%% however, it does increase the number of states in each individual process, and
%% so increases the compilation time.

The Scala code uses as assertion.  This is translated into a process that
diverges if the property does not hold.  In
Section~\ref{sec:syncchan-analysis-1}, we use FDR to verify that such
divergences do not occur.

Each operation is framed by events that represent the operations being called
or returning:
%
\begin{itemize}
\item The event \CSPM{beginSend.t.x} represents a thread with
  identity~\CSPM{t} calling |send(x)| on the channel, and the event
  \CSPM{endSend.t.SendSuccess} represents it successfully returning (later we
  add events to model unsuccessful calls that find the channel has
  been closed).

\item Likewise the events \CSPM{beginReceive.t} and
  \CSPM{endReceive.t.ReceiveSuccess.x} represent thread~\CSPM{t} calling and
  returning from an execution of the |receive| operation that successfully
  receives the value~\CSPM{x}.
\end{itemize}
%
We refer to these collectively as \CSPM{begin} and \CSPM{end} events.

%
%% Each thread is modelled by a process |Thread| that accepts a |begin| event,
%% performs the relevant operation, and then signals its completion via an |end|
%% event. 

The |ChannelThread| process in Figure~\ref{fig:CSP-send-receive} accepts the
\CSPM{begin} events, simulates the operation, and then performs the
appropriate \CSPM{end} event.

We construct the model of a channel, as in Figure~\ref{fig:basic-CSP-model}.
This model is encapsulated inside a CSP module, so that only the |begin| and
|end| events are visible outside the module.  A process outside the module can
simulate calling the operation by synchronising on the |begin| and |end|
events; later, we capture correctness properties in terms of those events.


%% We achieve this encapsulation by including inside the module a
%% process |Thread(t)| corresponding to each thread~|t|; this process accepts the
%% \CSPM{begin} events, simulates the operation, and then performs the
%% appropriate \CSPM{end} event.  
%% \begin{cspm}
%% Thread(me) = 
%%   beginSend.me?x -> Send(me, x); Thread(me)
%%   [] beginReceive.me -> Receive(me); Thread(me)
%% \end{cspm}

%% Figure~\ref{fig:send-scala-csp} gives the
%% outline of the |Thread| process, part of the Scala definition of the |send|
%% function, and the corresponding part of the CSP model.  The Scala |send|
%% function operates under mutual exclusion on~|lock| (denoted
%% ``|lock.mutex{...}|''); the corresponding parts of the CSP model call the
%% |Lock| and |Unlock| functions on the monitor.  The body of |send| starts
%% by waiting, on condition |slotEmptied|, until |status| is |Empty|; the process
%% |Send1| gives the corresponding CSP\null.  

Appendix~\ref{app:modelling} gives techniques for improving the structure of
such models.  We do not employ these techniques here, in the interests of
brevity, and because they are tangential to the main ideas of this paper.

%

%%%%%

%% \begin{figure}
%% \begin{center}
%% \begin{minipage}[t]{52mm}
%% \begin{scala}
%% def send(x: A) = lock.mutex{
%%   while(status != Empty) 
%%     slotEmptied.await()
%%   ...
%% }
%% \end{scala}
%% \end{minipage}
%% %
%% \qquad
%% %
%% \begin{minipage}[t]{100mm}
%% \begin{cspm}
%% Thread(me) = 
%%   beginSend.me?x -> Send(me, x)
%%   [] beginReceive.me -> Receive(me)
%% Send(me, x) = 
%%   Monitor::Lock(t); Send1(me, x); 
%%   Monitor::Unlock(t); endSend.me.SendSuccess -> Thread(me)
%% Send1(me, x) = 
%%   getStatus.me?s -> 
%%   if s == Empty then Send2(me, x)
%%   else (Monitor::Await(me, SlotEmptied); Send1(me, x))
%% Send2(me, x) = ...
%% \end{cspm}
%% \end{minipage}
%% \end{center}
%% \caption{Part of the Scala code for the {\scalastyle send} operation (left),
%%   and the corresponding CSP (right).}
%% \label{fig:send-scala-csp}
%% \end{figure}

%% This encapsulation  allows us to hide the internal events.  A process
%% outside the module can simulate calling the operation by synchronising on the
%% |begin| and |end| events; below, we capture correctness properties in terms of
%% those events.

%%%%%

%% Inside the channel module is a process corresponding to each thread, which
%% accepts the \CSPM{begin} events, simulates the operation, and then performs
%% the appropriate \CSPM{end} event.  The module encapsulates this so only the
%% \CSPM{begin} and \CSPM{end} events are visible.  A process outside the module
%% can simulate calling the operation by synchronising on those events.

