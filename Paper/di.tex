\section{On the number of data values}
\label{sec:di}

In this section we show that for each of the checks we have considered in this
paper (with a particular choice of threads), if the check succeeds when we use
a type |Data| of data values of size~2, then the check would also succeed for
a larger choice of~|Data|.  We consider |Data| a type parameter that can be
instantiated in different ways.
%% , and will
%% write $System[\CSPMM{Data}]$ to indicate that the family of $System$ processes
%% depend on this parameter. 

%%%%%%%%%%%%%%%%%%%%%%%%%%%%%%%%%%%%%%%%%%%%%%%%%%%%%%%

\subsection{Data independence}

Our approach makes use of techniques from data independence
\cite{Wolper-1986}\linebreak[1]\cite{ranko-thesis}\linebreak[1]%
\cite[Section~15.2]{awr:TPC}.  Consider a family $P[t]$ of processes,
parameterised by a type variable~$t$ that can be instantiated with different
non-empty types.  We say that the processes are \emph{data independent} in~$t$
if they can input, nondeterministically choose, store, and output values
from~$t$, but can't perform any other operations on such values, including
equality tests, and don't use any constants from~$t$.

The processes we have used in this paper are nearly data independent
in~|Data|, but with one exception. The implementation of |sendWithin(d)(x)|,
in the case of a time out, checks, via an assertion, that the |value| variable
still holds the value~|x| that the operation wrote there previously.  This
assertion plays no role in the functionality of the channel: it is merely a
sanity check.  But this assertion contains an equality test, so means the
process isn't data independent.  We could adapt the models, to make them data
independent, by simply removing this assertion from the model (and, indeed,
from the implementation).  For the moment, we consider these adapted models.
In Section~\ref{sec:di-extensions}, we argue that, in fact, this change isn't
necessary.

Previous versions of the models were not data independent for another reason.
The implementations of channels include a variable~|value| that stores the
current (or previous) value being sent; this variable is initialised
arbitrarily (to |null|).  This variable is modelled by a CSP process;
previously that process was initialised to a constant value~|A| from~|Data|.
Similarly, the model of an alt contains a process that models a variable that
holds the current (or previous) value received; this was also initialised to a
constant value.  This usage of constants from |Data| took the models outside
the realm of the data independence assumptions.  We therefore adapted the
models so that the initial values are chosen nondeterministically, so as to
make them data independent.  (This change made negligible difference to the
checks in Section~\ref{sec:syncchan}, but increased the state space of the
checks in Section~\ref{sec:alt} by 2--3\%.)

We will make use of the following results.
Consider a data independent family of processes $P[t]$.  Let $T$ and~$T'$ be
two concrete types, and let $f: T \fun T'$ be a surjective function.  Lift $f$
to events by pointwise application; and then lift to traces by pointwise
application.  The following results link the semantics of $P[T]$
and~$P[T']$~\cite{gavin:di}. 
%
\begin{itemize}
\item If $tr \in traces(P[T])$ then $f(tr) \in traces(P[T'])$;

\item If $(tr,X) \in failures(P[T])$ then\footnote{The notation
  $f\inverse\banana{e}$ represents the relational inverse image of~$e$
  under~$f$, i.e.~$\set{x \| f(x) = e}$.}
$(f(tr), \set{e \|  f\inverse\banana{e} \subseteq X}) \in failures(P[T'])$;


\item If $tr \in divs(P[T])$ then $f(tr) \in divs(P[T'])$. 
\end{itemize}
%
Informally, whenever $P[T]$ has a transition (in the operational semantics)
labelled with event~$e$,\, $P[T']$ has a corresponding transition labelled
with~$f(e)$, and vice versa; the above results lift this to the denotational
models.

%%%%%%%%%%%%%%%%%%%%%%%%%%%%%%%%%%%%%%%%%%%%%%%%%%%%%%%

\subsection{Synchronisation progressibility for basic channels}

Let $System[t]$ be one of the families of systems for channels that we
considered in Section~\ref{sec:syncchan}, where the type parameter~$t$
represents the type of data.  Let $T$ be an arbitrary type, with $\#T \ge 2$,
and let $T_2 = \set{A,B}$.  We show that if $System[T]$ fails synchronisation
progressibility, then so does $System[T_2]$; thus it is enough to verify
$System[T_2]$.

We start by considering basic channels, with just send and receive
operations. 

Consider a maximal failure $(tr,X)$ (i.e.~where $X$ contains \emph{all} events
that would be refused in the relevant state) of a correct implementation.
Suppose there is a pending \CSPM{send} operation (i.e., that has been called
but not returned); if that operation has synchronised, then the corresponding
|endSend| event is not refused (i.e.~is not in~$X$); and if the operation has
not synchronised, then the corresponding |endSend| event is refused.
Likewise, suppose there is a pending |receive| operation; if that operation
has synchronised with an invocation of |send(v)|, then the corresponding
|endReceive| event with data value~|v| is not refused (but all the other
corresponding |endReceive| events of that thread are refused); and if the
operation has not synchronised, then all corresponding |endReceive| events of
that thread are refused.  Thus by examining this maximal failure, we can
calculate which operations have synchronised.  This prompts the following
definition.
%
\begin{definition}
\label{def:synchronised}
Let $(tr,X)$ be a maximal failure of the system.  We say that an invocation
has \emph{synchronised} if either there is a corresponding |end| event with
a |Success| value (i.e.~the invocation returned successfully) or such an
|end| event is not refused.

The data value of an invocation \CSPM{send}$(v)$ is~$v$.  The data value of a
synchronised invocation of |receive| is either the data value in the
|endReceive| event, or (for a pending invocation) the data value in the
available |endReceive| event.
\end{definition}

\begin{definition}
\label{def:project}
We define the restriction of trace $tr$ to data value~$v$, written $tr
\project v$, to be the subtrace of $tr$ containing: (1)~events of invocations
whose data value is~$v$, and (2)~the |beginReceive| events of unsynchronised
invocations of |receive|.  We define the restriction of $(tr,X)$ to~$v$,
written $(tr,X) \project v$, to be $(tr \project v, X)$.
\end{definition}

%%%%%

The following lemma shows that, in deciding synchronisation progressibility,
we can consider each value in turn.  The contrapositive shows that if a
behaviour is not synchronisation progressible, there is a particular value
that's responsible.

\begin{lemma}
\label{lem:progressibility-specific-value}
Let $(tr,X)$ be a maximal failure.  Suppose that for every data value~$v$,\,
the restriction $(tr,X) \project v$ is synchronisation progressible;
then the whole failure is synchronisation progressible.
\end{lemma}

%%%%%

\begin{proof}
Consider a particular value of~$v$.  The fact that $(tr,X) \project v$ is
synchronisation progressible implies that there is some way of pairing off
the synchronised invocations for~$v$, corresponding to the synchronisations,
and such that no further such synchronisations are possible:
%
\begin{itemize}
\item If there is an unsynchronised invocation of $\CSPMM{send}(v)$, then
  there is no unsynchronised invocation of |receive|;

\item If there is an unsynchronised invocation of |receive|, there is no
  unsynchronised invocation of $\CSPMM{send}(v)$.
\end{itemize}

Considering all values of~$v$ together, we deduce that there is some way of
pairing off all the synchronised invocations, corresponding to the
synchronisations, and such that no further such synchronisations are possible:
%
\begin{itemize}
\item If there is an unsynchronised invocation of |send|, then
  there is no unsynchronised invocation of |receive|;

\item If there is an unsynchronised invocation of |receive|, there is no
  unsynchronised invocation of |send|.
\end{itemize}
%
This implies that the whole failure is synchronisation progressible. 
\end{proof}

%%%%%

\begin{note}
The above lemma does not hold when we replace synchronisation progressibility
(a property of stable failures) by synchronisation linearisability (a property
of traces).  Consider the trace:
\begin{eqnarray*}
tr &  =  &
  \trace{ \CSPMMR{beginSend.t1.A},\, \CSPMMR{beginSend.t2.B},\, 
    \CSPMMR{beginReceive.t3},\, \CSPMMR{endSend.t1},\, \CSPMMR{endSend.t2} }.
\end{eqnarray*}
This represents a complete invocation of |send(A)|, a complete invocation of
|send(B)|, and a pending invocation of |receive|, all overlapping.  For each
value~$v$, the restriction $tr \project v$ is synchronisation linearisable:
the |send| could synchronise with the pending |receive|.  However, $tr$ itself
is clearly not synchronisation linearisable, because only one |send| can
synchronise with the |receive|.
\end{note}

%%%%%%%%%%

\begin{prop}
Consider some type $T$ of data values, with $\#T \ge 2$.  Suppose $System[T]$
is not synchronisation progressible.  Then $System[T_2]$ is not
synchronisation progressible.
\end{prop}

\begin{proof}
By the assumption, there is some maximal failure $(tr,X)$ of $System[T]$ that
is not synchronisation progressible.  Then, by
Lemma~\ref{lem:progressibility-specific-value}, there is a value~$v$ such that
$(tr,X) \project v$ is not synchronisation progressible.
%
Consider the function $f: T \fun T_2$ such that $f(v) = A$, and $f(x) = B$ for
all $x \ne v$.  Then 
\begin{eqnarray*}
(f(tr), \set{e \in \Sigma \| f\inverse(e) \subseteq X}) & \in &
   failures(System[T_2]),
\end{eqnarray*}
by the earlier data-independence result.  But that failure is not
synchronisation progressible for the same reason that $(tr,X) \project v$ is
not synchronisation progressible.
\end{proof}


%%%%%%%%%%%%%%%%%%%%%%%%%%%%%%%%%%%%%%%%%%%%%%%%%%%%%%%

\subsection{Extensions}
\label{sec:di-extensions}

We now extend our analysis to consider the closing of channels, the timed
operations and alts.  We then consider divergence freedom, and remove the
restriction mentioned earlier.

%%%%%

\paragraph{Closing and timed operations.}

We extend Definition~\ref{def:synchronised} to |sendWithin| and
|receiveWithin| in the obvious way.  Given a maximal failure, we say that an
invocation of |send|, |receive|, |sendWithin| or |receiveWithin| \emph{detects
  closure} if either it returns the |Closed| value, or is pending and the
return of the |Closed| value is not refused (so all successful returns are
refused).  We say that a |sendWithin| or |receiveWithin| invocation
\emph{times out} if either it returns |Timeout|, or such an event is not
refused.

We extend the definition of the restriction of trace $tr$ to data value~$v$,
written $tr \project v$, to include: (1)~events of invocations whose data
value is~$v$; (2)~the |begin| events of unsynchronised invocations of
|receive| and |receiveWithin|; (3)~|begin| and |end| events of all invocations
that detect closure; (4)~|begin| and |end| events of all invocations that
time out; and (5)~all |beginClose| and |endClose| events.

The proof then proceeds much as before.  In the proof of the adaptation of
Lemma~\ref{lem:progressibility-specific-value}, the fact that $(tr,X) \project
v$ is synchronisation progressible implies there is some way of choosing the
first linearisation point of an invocation of |close| (if there is one) such
that:
%
\begin{itemize}
\item all invocations that synchronise with data value~$v$ can be linearised
  before the linearisation of |close|;

\item all invocations (for any data value) that time out can be linearised
  before the linearisation of |close|;

\item all invocations (for any data value) that detect closure can be
  linearised after the linearisation of |close|.
\end{itemize}
%
Consider all values of~$v$ together.  Let $\hat{v}$ be the value of~$v$ that
has the latest linearisation point for |close|; we linearise the |close| at
this point.
%
\begin{itemize}
\item This is necessarily after all linearisations of synchronisations and
  invocations that time out, by the first two bullet points above;

\item The |begin| and |end| events of all invocations that detect closure are
  included in $(tr,X) \project \hat{v}$; hence they are after the 
  linearisation point of |close|, by the third bullet point above.
\end{itemize}

%%%%%%%%%%%%%%%%%%%%%%%%%%%%%%%%%%%%%%%%%%%%%%%%%%%%%%%

\paragraph{Alts.}

We now extend the analysis to alts.  Consider a maximal refusal.  We say that
an invocation of the alt \emph{synchronises to receive~$x$} if either an
|endAlt.t.AltReceive.i.|$x$ event occurs, or such an event is not refused.  We
say that an invocation \emph{synchronises to send~$x$} similarly.  We say that
an invocation \emph{aborts} if either an |endAlt.t.AltAbort| event occurs, or
such an event is not refused.

We then extend the restriction of trace~$tr$ to data value~$v$ in the obvious
way, to also include: (1)~|beginAlt| and |endAlt| events of invocations that
send or receive~$v$; and (2)~|beginAlt| events of invocations that fail to
synchronise (so either abort or are blocked).

The proof then proceeds much as before.

%%%%%%%%%%%%%%%%%%%%%%%%%%%%%%%%%%%%%%%%%%%%%%%%%%%%%%%

\paragraph{Divergence freedom.}

Suppose  for some concrete type $T$, with $\#T \ge 2$, that $System[T]$
can diverge after trace~$tr$.  Let $f : T \fun T_2$ be an arbitrary
surjection.  Then by the earlier data-independence result, $f(tr)$ would be a
divergence for $System[T_2]$.  Thus it is enough to verify that $System[T_2]$
is divergence-free.  (In fact, we could consider a single data value here; but
we need two data values for the argument in the following paragraphs.)

Recall that the implementation of |sendWithin| uses an equality test,
corresponding to an assertion, and so the model is not data independent: if
the equality test returns false, the process diverges.  We now argue that when
we include this assertion in the model, verifying there is no such divergence
when using $T_2 = \set{A, B}$ implies that there is no such divergence for
larger types of data.

Suppose otherwise, and there is a divergence in the system using~$T$, say when
the process compares values~$x$ and~$y$, following trace~$tr$.  Consider the
function~$f$ that maps $x$ to~$A$, and maps all other values to~$B$.  Then the
system using~$T_2$ would compare values~$A$ and~$B$ following trace~$f(tr)$,
and so would also have a divergence.  This is a contradiction.

Thus, the behaviours of the models with and without the assertion have
identical behaviours.  In particular, the former satisfies synchronisation
progressibility, because the latter does. 

