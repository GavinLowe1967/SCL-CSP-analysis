%%% Text common to the "idealised channel" section, in both the TR and the
%%% submitted version.

We can then compare the CSP model of a synchronous channel implementation
against the idealised model.  We create a system using the model of the
implementation, allowing threads to call appropriate operations.  We can then
verify that this system refines |ChannelSpec|$_F$ and |ChannelSpec|$_{FD}$ in
the relevant models.

\begin{window}[1,r,{
%
\vspace{0.5ex}
\begin{tabular}{\|cccc\|}
Model & Threads & States  & Time\\
F & 4 & 236M & 569s \\
FD & 4 & 176M & 107s \\
\end{tabular}
},]
%
The table to the right gives statistics about the number of states explored
and the times taken to perform these checks, in different models; in each
case, one thread was an alt-thread and the remainder where channel-threads.
With five threads, the checks fail to complete on the available architecture.
\end{window}

The bottleneck in these checks is the time taken to normalise the
specification: this accounts for about 90\% of the total in the
stable-failures model; checks with more than four threads get stuck at this
point.  This step builds an automaton equivalent to the specification, but
with the property that after each trace (of visible events), a unique state is
reached; if, after trace~$tr$, the specification can reach
states~$st_1,\ldots,st_k$, then the normalised automaton contains a
corresponding state equivalent to $st_1 \mathbin{\CSPMM{\|~\|}} \ldots
\mathbin{\CSPMM{\|~\|}} st_k$.  The idealised channel is a complex process,
with much internal nondeterminism, so normalising it is slow.

Interestingly, the failures-divergences check is faster than the
stable-failures model: normally, it is the other way round.  The difference is
due to the time taken to normalise the specification.
Consider the case where a trace~$tr$ might have included a (hidden) error
event.  In the failures-divergences model, the resulting normalised state
after~$tr$ includes |DIV|, and so is equal to~|DIV|: the normalisation
algorithm identifies this, and so does not need to expand its successor
states.  However, in the stable-failures model, the algorithm does not
identify that the corresponding state is equivalent to |CHAOS(Interface)|: it
does construct the successor states, making normalisation much slower than in
the failures-divergences model.  It is straightforward to show that the
failures-divergences check implies the corresponding stable-failures check,
because |IdealisedChan| is divergence-free.


