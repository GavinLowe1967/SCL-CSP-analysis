\subsection{Idealised model of a channel}
\label{sec:idealisedChan}

%\cspMid

We now sketch the idealised model of a channel: this describes the behaviour
of a channel in terms of the calls and returns of operations, while
abstracting away from the implementation.  We assume (for simplicity) a single
thread, |AltThread|, that runs any alt that interacts with the channel.
%
The alphabet of the idealised channel is the same as the channel in
Figure~\ref{fig:alt-chans}, together with error events representing erroneous
registrations and deregistrations.

The model keeps track of whether the channel is open or closed.  However, the
closing is wrapped between |beginClose| and |endClose| events.  The closing
happens at a nondeterministic point between these events: we say the
operation is \emph{linearised} at that point.

The model keeps track of which threads are trying to send or receive on the
channel, updating its state as a result of the relevant |begin| events.  It
also keeps track of which such threads have synchronised and so can return
(indicated by |end| events).  If one thread is trying to send, and another is
trying to receive, then they can synchronise, and both become able to return.
Alternatively, a timed send or receive can nondeterministically timeout, and
so become able to return with a negative result.  If the channel is closed,
all such pending operations become able to return with a |Closed| result.

The model also keeps track of whether an alt is registered at a port.  Each
registration is linearised between the corresponding |begin| and |end|
events.  There are several possibilities:
%
\begin{itemize}
\item If a thread is waiting, trying to send on the channel when a
  registration happens at the in-port, then the two can synchronise: the send
  operation becomes able to return, and the registration returns with a
  successful result.

\item The case of a synchronisation between a waiting receive operation and a
  registration at the out-port is similar.

\item If no such synchronisation is possible, the registration is recorded,
  and the registration can return with a negative result.

\item If the channel is closed, the registration can return with a
  |RegisterClosed| result. 

\item If a registration happens when an alt is already registered at either
  port, an error event is performed.
\end{itemize}

Deregistrations are again linearised between the corresponding |begin| and
|end| events.  If an alt is registered with the channel, and a deregistration
happens for a \emph{different} alt (or with the wrong index), then an error
event is performed (however, it's fine to allow a deregistration when no port
is registered, as this can happen in the implementation, if a channel thread
has performed a callback concurrently with the deregistration phase: the
subsequent deregistration is just ignored).

It an alt is registered at the in-port, and a thread tries to send on the
channel, then the channel may attempt a callback, via a |beginMaybeReceive|
event.  No other operation can be linearised until after a corresponding
|endMaybeReceive| event (this reflects the fact that the channel is locked
during this period).  If the callback is successful, the sending operation
becomes able to return.  The case of an alt being registered at the out-port
when a thread tries to receive on the channel is similar.

If an alt is registered at a port, and the channel is closed, the idealised
model performs a callback, represented by |beginPortClosed| and
|endPortClosed| events.

As can be seen by the above discussion, the model is rather complex.  In
particular, if there are multiple synchronisations or linearisations that can
occur, the model allows any of them to occur, nondeterministically.  

The model is structured as the parallel composition of three families of
processes: one process for each channel thread, modelling and linearising that
thread's operations; one process keeping track of the state of the channel;
and one linearising and keeping track of registrations and deregistrations by
the alt.  These processes synchronise together to achieve the above effect.
This structure makes it easier to allow all the desired interleavings of
events, in particular to ensure operations are linearised between the
corresponding |begin| and |end| events.  See~\cite{TR} for full details.

This produces a process |IdealisedChannel|.  Recall that we want to allow
arbitrary behaviour if the environment has not followed the alt protocol,
represented by events from a set |Errors|$_C$.  
%% In the stable-failures model, the process |CHAOS(Interface)| allows
%% arbitrary behaviour over the set |Interface| (the interface of the
%% channel); in the failures-divergences model, the process |DIV| allows
%% arbitrary behaviour.
We therefore define the following, corresponding to the failures and
failures-divergences models.
\begin{cspm}
ChannelSpec£$_F$£ = (IdealisedChannel [|Errors£$_C$£|> CHAOS(Interface) ) \ Errors£$_C$£
ChannelSpec£$_{FD}$£ = (IdealisedChannel [|Errors£$_C$£|> DIV ) \ Errors£$_C$£
\end{cspm}

%%% Text common to the "idealised channel" section, in both the TR and the
%%% submitted version.

We can then compare the CSP model of a synchronous channel implementation
against the idealised model.  We create a system using the model of the
implementation, allowing threads to call appropriate operations.  We can then
verify that this system refines |ChannelSpec|$_F$ and |ChannelSpec|$_{FD}$ in
the relevant models.

\begin{window}[1,r,{
%
\vspace{0.5ex}
\begin{tabular}{\|cccc\|}
Model & Threads & States  & Time\\
F & 4 & 236M & 569s \\
FD & 4 & 176M & 107s \\
\end{tabular}
},]
%
The table to the right gives statistics about the number of states explored
and the times taken to perform these checks, in different models; in each
case, one thread was an alt-thread and the remainder where channel-threads.
With five threads, the checks fail to complete on the available architecture.
\end{window}

The bottleneck in these checks is the time taken to normalise the
specification: this accounts for about 90\% of the total in the
stable-failures model; checks with more than four threads get stuck at this
point.  This step builds an automaton equivalent to the specification, but
with the property that after each trace (of visible events), a unique state is
reached; if, after trace~$tr$, the specification can reach
states~$st_1,\ldots,st_k$, then the normalised automaton contains a
corresponding state equivalent to $st_1 \mathbin{\CSPMM{\|~\|}} \ldots
\mathbin{\CSPMM{\|~\|}} st_k$.  The idealised channel is a complex process,
with much internal nondeterminism, so normalising it is slow.

Interestingly, the failures-divergences check is faster than the
stable-failures model: normally, it is the other way round.  The difference is
due to the time taken to normalise the specification.
Consider the case where a trace~$tr$ might have included a (hidden) error
event.  In the failures-divergences model, the resulting normalised state
after~$tr$ includes |DIV|, and so is equal to~|DIV|: the normalisation
algorithm identifies this, and so does not need to expand its successor
states.  However, in the stable-failures model, the algorithm does not
identify that the corresponding state is equivalent to |CHAOS(Interface)|: it
does construct the successor states, making normalisation much slower than in
the failures-divergences model.  It is straightforward to show that the
failures-divergences check implies the corresponding stable-failures check,
because |IdealisedChan| is divergence-free.




%%%%%%%%%%%%%%%%%%%%%%%%%%%%%%%%%%%%%%%%%%%%%%%%%%%%%%%

\subsection{Idealised model of an alt}
\label{sec:idealisedAlt}

We now describe the construction of the idealised model of an alt.  
%
The alphabet of the idealised alt is the same as the alt in
Figure~\ref{fig:alt-chans}, together with error events representing erroneous
callbacks that do not correspond to any current registration.

The model responds to a |beginAlt| event by registering, in turn, at the ports
corresponding to its branches.  It performs a |beginRegister| event for each,
and waits to receive back an |endRegister| event; it keeps track of the
results.  If it receives a successful result, it moves onto the deregistration
phase.  If it completes the registration and finds every channel was closed,
it returns an |AltAbort| result on |endAlt|.  Otherwise, it moves to the
waiting phase.

The model allows callbacks; each callback is linearised between the
corresponding |begin| and |end| events.  Callbacks cannot be linearised during
the registration phase (in the implementation, the alt-thread holds the alt's
lock during this phase).  A callback of |maybeSend| or |maybeReceive| can be
linearised successfully during the waiting phase, if it corresponds to a valid
registration; this moves the model to the deregistration phase.  A callback of
|portClosed| can also be linearised; if this was for the final open channel,
the alt reacts with a |AltAbort| result on |endAlt|, and returns to the initial
state.  However, a callback that does not correspond to a correct registration
is signalled via an error event.

In addition, during the waiting phase, the alt might signal a spurious
wakeup. 

During the deregistration phase, the idealised model deregisters from each of
its registered ports (via appropriate |begin| and |end| events).  If further
callbacks occur, these are linearised with a negative result.  When all
deregistrations are complete, the model signals with an appropriate |endAlt|
event.

If a callback occurs outside of the execution of the alt (e.g.~before the
first |beginAlt| event), this causes an error event. 

As with the idealised channel, the idealised alt is defined as the parallel
composition of three families of processes: one process for each channel
thread, modelling and linearising that thread's callbacks; a process that
keeps track of the current registrations; and a process that controls the
overall execution.  See~\cite{TR} for  full details.

The above procedure produces a process |IdealisedAlt|.  Let |Errors|$_A$ be
the corresponding error events.  We then allow arbitrary behaviours after the
error events, much as for the idealised model of a channel.
%
\begin{cspm}
AltSpec£$_F$£ = (IdealisedAlt [|Errors£$_A$£|> CHAOS(Interface) ) \ Errors£$_A$£
AltSpec£$_{FD}$£ = (IdealisedAlt [|Errors£$_A$£|> DIV ) \ Errors£$_A$£
\end{cspm}

%%% Text from the "idealised alt" section, common to both the TR and the
%%% submitted version. 

We can then compare the CSP model of an alt implementation to the idealised
model.  We can verify that the implementation model refines |AltSpec|$_F$ and
|AltSpec|$_{FD}$ in the relevant semantic models.

\begin{window}[1,r,{
%
\vspace{0.5ex}
\begin{tabular}{\|cccc\|}
Model & Threads & States  & Time\\
F & 4 & 33.1M & 21s \\
FD & 4 & 5.6M & 5.4s \\
F & 5 & 601M & 472s \\
FD & 5 & 62.7M & 46s \\
F & 6 & 10.6B & 153min \\
FD & 6 & 662M & 701s \\
\end{tabular}
},]
%
The table to the right gives statistics about these checks.  In each case, we
considered an alt with two branches, one in-port branch and one out-port
branch.  The checks are faster than the corresponding checks for a channel,
mainly because normalisation of the specification was
faster than for a channel, because the idealised alt has fewer states than an
idealised channel.  We think the main reason for this is that the |Lin(t)|
processes in the idealised alt have fewer states than the |Lin(t)| processes
in the idealised channel (there is one such process for each channel thread,
so the total number of states increases exponentially).  In addition, the
implementation model for an alt has fewer states than for a channel.
\end{window}

We used a couple of techniques to improve the efficiency of these checks.
Firstly, we applied the prioritisation operator of FDR to |IdealisedAlt| to
give priority to error events over all other events (except $\tau$s); so in
any state where an error event (or $\tau$) is possible, all other events are
blocked.  This is sound here because the subsequent behaviours allowed by the
specification (in |CHAOS(Interface)| or |DIV|) include the behaviours
corresponding to the blocked events.

Further, for the check in the stable-failures model, we normalised
|IdealisedAlt| before the application of the throws operator.  This made the
subsequent normalisation of |AltSpec|$_F$ faster.  (However, the same
technique in the failures-divergences model made the check slower.)



