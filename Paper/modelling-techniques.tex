\section{Modelling techniques}

\inlineCSP

We describe here a few modelling techniques that we found useful during our
analysis.  These techniques are rather orthogonal to the main ideas of the
paper, but we think they might be useful elsewhere. 

We model variables using a CSP process as follows.  Here |value| is the
current value of the variable, and |get| and |set| are channels on which a
thread~|t| can get or set the value.
%
\begin{cspm}
Var(value, get, set) = 
  get?t!value -> Var(value, get, set)
  [] set?t?value' -> Var(value', get, set)
\end{cspm}
%
For example, the implementation of a channel has a variable
\SCALA{receiversWaiting} that stores the current number of threads that are
waiting to receive on the channel.  In the implementation, this can be an
arbitrary \SCALA{Int}; however, we would expect the value to be non-negative
and at most the number of threads in the CSP model (and the subsequent
analysis confirms this).  The variable can therefore be modelled as follows. 
%
\begin{cspm}
N = card(ThreadID)
channel getReceiversWaiting, setReceiversWaiting : ThreadID.{0..N}
ReceiversWaiting = Var(0, getReceiversWaiting, setReceiversWaiting)
\end{cspm}

The Scala implementation contains a number of assertions.  We model such
assertions by having the model diverge if the property does not hold; our
subsequent analysis verifies that such a divergence does not happen.  (An
alternative is to perform an explicit |error| event; however, our experience
is that it is easy to get that approach wrong, for example by accidentally
omitting |error| from a process's alphabet.)  The following two macros are
useful.
%
\begin{cspm}
Assert(b, P) = if b then P else DIV
Assert1(b) = Assert(b, SKIP)
\end{cspm}
%
In the former, the continuation |P| is provided explicitly, whereas the latter
can be used with sequential composition.  The two processes \CSPM{Assert(b,
  P)} and \CSPM{Assert1(b); P} are CSP-equivalent; but the FDR compiler treats
them differently.  In the former, |P| is compiled only if~|b| is true, but in
the latter, |P| is compiled regardless.  This can make a difference if
compilation would fail when |b| is false.  The following macros, to increment
or decrement the \SCALA{receiversWaiting} variable, illustrate this point; the
use of |Assert| ensures that the compiler does not try to produce an event on
|setReceiversWaiting| outside the correct range |{0..N}|.
%
\begin{cspm}
IncReceiversWaiting(me) = 
  getReceiversWaiting.me?r -> Assert(r < N, setReceiversWaiting.me.r+1 -> SKIP)
DecReceiversWaiting(me) =
  getReceiversWaiting.me?r -> Assert(r > 0, setReceiversWaiting.me.r-1 -> SKIP)
\end{cspm}

We now discuss the modelling of Scala functions in the implementation, in
particular how to model the value returned.  CSP has no notion of returning a
value.  Instead, we adopt a continuation-passing approach.  If the Scala
function returns a result of type~\SCALA{A}, the corresponding CSP process is
given a parameter \CSPM{cont :: A -> Proc},  representing a continuation,
i.e.~what the rest of the program does with the result.  Returns from the
Scala function are then modelled by applying |cont| to the returned value.  A
call to the function is modelled by providing a suitable function as the
continuation.

A continuation-passing style can also be used to model computations that are
passed to other constructs.  For example, the following macro captures an
\SCALA{if} statement.  The test of the \SCALA{if} is modelled by a process
|test(k)| that applies its continuation parameter~|k| to an appropriate
boolean. 
%
\begin{cspm}
If:: (((Bool) -> Proc) -> Proc, Proc, Proc) -> Proc
If(test, P, Q) = test(£$\lambda$£res @ if res then P else Q)
\end{cspm}
%
For example, Scala code \SCALA{if(result == None && status != Filled)\{...\}
  else \{...\}} can be modelled as follows. 
%
\begin{cspm}
let Test(k) = if result==None then getStatus.me?s -> k(s != Filled) else k(false)
within If(Test,...,...)
\end{cspm}
%
Likewise, a \SCALA{while} loop can be captured using the following macro.
%
\begin{cspm}
While :: (((Bool) -> Proc) -> Proc, Proc) -> Proc
While(test, body) = test(£$\lambda$£b @ if b then body; While(test, body) else SKIP)
\end{cspm}
