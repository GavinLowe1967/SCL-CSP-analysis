\section{Overview of CSP} 
\label{sec:csp}

In this section we give a brief overview of the syntax for the fragment of CSP
that we will be using in this paper.  We then review the relevant aspects of
CSP semantics, and the use of the model checker FDR in verification.  For more
details, see~\cite{awr:UCS}.

CSP is a process algebra for describing programs or {\em processes}\/ that
interact with their environment by communication.  Processes communicate via
atomic \emph{events}.  Events often involve passing values over channels; for
example, the event \CSPM{c.3} represents the value~\CSPM{3} being passed on
channel~\CSPM{c}.  Channels may be declared using the keyword \CSPM{channel};
for example, \CSPM{channel c : Int} declares \CSPM{c} to be a channel that
passes an \CSPM{Int}.  (In this paper, the word ``channel'' can mean either an
SCL channel or a CSP channel; the intention should be clear from the context.)
The notation \CSPM{\{|c|\}} represents the set of events over
channel~\CSPM{c}.

The simplest process is \CSPM{STOP}, which represents a deadlocked process that
cannot communicate with its environment.  The process |SKIP| is a process that
terminates immediately, represented by the distinguished event~$\tick$. 

The process \CSPM{CHAOS(A)} can perform any events from the set~|A|, or can
refuse any of those events; however, it cannot diverge.  Thus it allows
arbtrary non-divergent behaviour over~|A|.  By contrast, \CSPM{DIV} is a
divergent process that performs an unbounded number of internal $\tau$~events.

The process \CSPM{a -> P} offers its environment the event~\CSPM{a}; if the
event is performed, the process then acts like~\CSPM{P}.  The process
\CSPM{c?x -> P} is initially willing to input a value \CSPM{x} on
channel~\CSPM{c}, i.e.~it is willing to perform any event of the
form~\CSPM{c.x}; it then acts like~\CSPM{P} (which may use~\CSPM{x}).
Similarly, the process \CSPM{c?x:X -> P} is willing to input any
value~\CSPM{x} from set~\CSPM{X} on channel~\CSPM{c}, and then act
like~\CSPM{P}.  Within input constructs, we use ``\CSPM{_}'' as a wildcard:
\CSPM{c?_} indicates an input of an arbitrary value.  The process \CSPM{c!v ->
  P} outputs value \CSPM{v} on channel~\CSPM{c}.  Inputs and outputs may be
mixed within the same communication, for example \CSPM{c?x!v -> P}.

The process \CSPM{P [] Q} can act like either \CSPM{P} or~\CSPM{Q}, the choice
being made by the environment: the environment is offered the choice between
the initial events of~\CSPM{P} and~\CSPM{Q}.  By contrast, \CSPM{P |~| Q} may
act like either~\CSPM{P} or~\CSPM{Q}, with the choice being made
nondeterministically, not under the control of the environment.  \CSPM{**[]
  x:X @ P(x)} is an indexed external choice, with the choice being made over
the processes \CSPM{P(x)} for~\CSPM{x} in~\CSPM{X}.  Likewise, 
\CSPM{**|~| x:X @  P(x)} represents a nondeterministic choice over the~\CSPM{P(x)}.

The process \CSPM{if b then P else Q} represents a conditional. 
The process \CSPM{b & P} is a guarded process, that makes \CSPM{P} available
only if \CSPM{b} is true; it is equivalent to \CSPM{if b then P else STOP}.

The process |P; Q| represents a sequential composition of~|P| and~|Q|:
initially, |P| is run, but when it terminates (as indicated by event~$\tick$),
|Q| is run. 

The process \CSPM{P [|E|> Q} 
% {]}
(sometimes denoted $\CSPMM{P} \mathbin{\theta_E} \CSPMM{Q}$) denotes a
  throw-catch mechanism: initially, |P| is executed, but if it performs an
  event from~|E|, control is passed to~|Q|: the events from~|E| can be thought
  of as exceptions, and |Q| as an exception handler.

The process \CSPM{P [|A|] Q} runs \CSPM{P} and~\CSPM{Q} in parallel,
synchronising on events from~\CSPM{A}.  The process \CSPM{**|| x:X @ [A(x)]
  P(x)} represents an indexed parallel composition, where, for each \CSPM{x}
in \CSPM{X},\, \CSPM{P(x)} is given alphabet~\CSPM{A(x)}; processes
synchronize on events in the intersection of their alphabets.  
% By contrast, in
% the process \CSPM{[|A|] x:X @ P(x)}, all processes \CSPM{P(x)} synchronize on
% events from~\CSPM{A}, but performs other events without synchronization. 
The process \CSPM{P ||| Q} interleaves \CSPM{P} and \CSPM{Q}, i.e.\ runs them
in parallel with no synchronisation.  \CSPM{**||| x:X @ P(x)} represents an
indexed interleaving.

The process \CSPM{P \\ A} acts like~\CSPM{P}, except the events from~\CSPM{A}
are hidden, i.e.~turned into internal $\tau$~events.

CSP, as implemented in the model checker FDR, is supported by a functional
sublanguage, roughly equivalent to Haskell, but without type classes, and
augmented with sets and mappings. 
 
A \emph{trace} of a process is a sequence of (visible) events that a process
can perform.  We say that \CSPM{P} is refined by~\CSPM{Q} in the traces model,
written \CSPM{P [T= Q}, if every trace of~\CSPM{Q} is also a trace
of~\CSPM{P}\@.  FDR can test such refinements automatically, for finite-state
processes.  Typically, \CSPM{P} is a specification process, describing what
traces are acceptable; this test checks whether \CSPM{Q} has only such
acceptable traces.  

Traces refinement tests can only ensure that no ``bad'' traces can occur: they
cannot ensure that anything ``good'' actually happens; for this we need the
stable failures or failures-divergences models.  A \emph{stable failure} of a
process~\CSPM{P} is a pair $(tr,X)$, which represents that \CSPM{P} can
perform the trace~$tr$ to reach a stable state (i.e.~where no internal events
are possible) where $X$ can be refused, i.e., where none of the events of~$X$
is available.  We say that \CSPM{P} is refined by~\CSPM{Q} in the stable
failures model, written \CSPM{P [F= Q}, if every trace of~\CSPM{Q} is also a
  trace of~\CSPM{P}, and every stable failure of~\CSPM{Q} is also a stable
  failure of~\CSPM{P}.

We say that a process \emph{diverges} if it can perform an infinite number of
internal (hidden) events without any intervening visible events.  We do not
use the full power of the failures-divergences model in this paper.  The only
failures-divergences refinement checks we use are of the form \CSPM{P [FD= Q}
  where |P| is divergence-free; this tests that |Q| is also divergence-free,
  and that \CSPM{P [F= Q}.


%% If  \CSPM{P [F= Q} and \CSPM{Q} is divergence-free, then if
%% \CSPM{P} can stably offer an event~\CSPM{a}, then so can~\CSPM{Q}; hence
%% such tests can be used to ensure \CSPM{Q} makes useful progress.


